\chapter{一次方程式}


    在第一章,复习小学算术的基础之上,又引进了
负数,建立了对于加、减、乘、除以及乘方运算通行
无阻的有理数系,还着重讨论了对任何有理数都适用
的“运算通性”.这样一来,当我们用字母表示数
时,就应该注意领会:这个字母既可以表示所指范围
内的任一个数,同时也具有数的运算通性这两层意
义;例如:矩形的面积公式$S=a\cdot b$中,字母$a,  b,  S$
分别表示矩形的长度、宽度与面积.它们可以表示所
有正数中的任一个;在它们的运算中,同样具有“运
算通性”.

    本章将在字母表示数的基础上,由应用问题入
手,引进一次方程式,并应用代数方法去讨论解应用
题的一般途经.

\section{算术解法与代数解法}
    这一节我们将以应用题为目标,分别用算术解法
与代数解法来解决,从中领会代数解法的要点,并通
过比较分析,初步认识代数解法的一普遍性与优越性.
进而去掌握代数解法的原理,熟练运用其方法解决应
用问题.

\subsection{两种解法的分析、对比}
从两个实例谈起:

\begin{example}
    某农场计划播种小麦与大豆共138亩,要求
种小麦的面积是大豆的四倍,试问:该农场应种小麦
与大豆各多少亩?
\end{example}

【算术解法分析】由题目所给条件,可知:播种
总面积就是种大豆面积的$(4 + 1)$倍.因此,
\[\begin{split}
    \text{(种大豆亩数)}&=\text{(总亩数)}\div (4 + 1)\\
    \text{(种小麦亩数)}&=\text{(总亩数)}-\text{(种大豆亩数)}
\end{split}\]
即:
\begin{align*}
    138\div(4 + 1)&=27.6\text{(亩)}\tag{大豆亩数}\\
    138-27. 6 &= 110.4\text{(亩)}\tag{小麦亩数}
\end{align*}

当然,求出种大豆的亩数以后,也可根据题目已
知,4倍这个数,就是种小麦亩数.

【代数解法分析】可以用一个字母$x$表示我们所
求的一个数量,例如在此题中,$x$可以表示“种大豆
的亩数”;再由题目所给条件,显然可知“种小麦的
亩数”就应该用$4x$表示;因此,只要根据“(总亩数)
$=$(种小麦亩数)$+$(种大豆亩数)”的关系式,和
题目给出“总亩数$=13$”的条件,就可以直截了当
地得到一个等式:
\[4x+x=138\]    

对于这个等式,比起算术中的算式来说,只是多
了一个“表示要求的数的字母$x$”,尽管还不知道$x$
究竟是多少,但终归是一个数,就必然要具有“数的
运算通性”,特别是在运算中同样可以有效地对它使
用加、乘的交换、结合和分配律;又因为所列是一个
等式,就必然具有等式的基本性质.因此,我们可以
作如下的变形:
\begin{align*}
    (4+1)x=138  \tag{分配律}
\end{align*}
即:$5x=138$
\begin{align*}
    \therefore\quad x&=138\div 5  \tag{等式两边同除以5}\\
    x&=27.6\text{(亩)} \tag{种大豆亩数}\\
    4x&=4\x 27.6=110.4\text{(亩)} \tag{种小麦亩数}
\end{align*}

【对比】算术解法中,要求对题意进行思考,说
明每一个算式的意义.如$(4+1)$表示“总亩数是种
大豆亩数”的倍数等;而代数解法只要求用字母$x$表
示所求的一个数量(例2.1中,可以自己练习设$x$表示
  “种小麦亩数”的解法),将$x$与已知的数量一起考
虑它们之间的关系,根据题意能够直接了当地把关系
列成一个等式然后应用“运算通性”及“等式性
质”求出$x$应有的值.

\begin{example}
    小明用六角四分人民币买面值为4分、8分
两种邮票一共12张.试问:小明买了4分、8分邮票各几张?
\end{example}

【算术解法分析】首先假定小明所买12张都是
9分邮票,就应该花钱$12 \x 0. 04=0.48$(元).而这
与实际所花的钱$0. 64$(元)还相差$(0. 64 - 0.48)
=0.16$(元).

    其次,为了使所花钱数增加,且邮票张数不变,
就要用8分邮票去换取4分邮票,每换取一张,钱数
增加$(0. 08-0. 04)=0. 04$(元),而邮票张数还保持
不变.这样,我们的问题就可以转化为:换几次(用
几张8分邮票)后,才能使所花钱数正好能补上所差
的0.16(元)呢?

    由此可以得出本题算术解法是:
\begin{equation}
    \begin{split}
 \text{“8分邮票的张数”}&=\text{“换取的次数”}   \\
 &=[0.64-(12\x 0.04)]\div (0.08-0.04) \\
 &=0.16\div 0.04=4\text{(张)}   
    \end{split}
\end{equation}
\[ \text{“4分邮票的张数”}=12-4=8 \text{(张)} \]

【代数解法分析】首先仍可以用一个字母$y$表
示所求的“8分邮票的张数”;由题目显然可以
知道“4分邮票的张数”就应该用$(12-y)$表示
了.

    其次,由于$\text{(每张邮票的价钱)}\x\text{(张数)}=\text{(这
种邮票所用的钱数)}$,因而可以知道小明买8分、4分
邮票分别使用的钱数为:$(0,08\cdot y)$元与$0.04(12-y)$元.

    再次,由于两种邮票总共花钱六角四分,因此就
得到:
\begin{equation}
    (0.08)y+0.04(12-y)=0.64
\end{equation}

最后,由(2.2)出发,运用数系运算通性和等式
的性质(特别是分配律),就可求出$y$应取的值来,
即:
\begin{align*}
    0.08y+0.04 \x 12-0.04y&= 0.64  \tag{分配律}\\
    (0.08y-0.04y)+0.48&=0.64          \tag{交换、结合律}\\
    (0.08-0.04)y+0.48&=0.64\tag{分配律}\\
    0.04y&= 0.64-0.48\tag{两边同减0.48}\\
        0.04y&=0.16
\end{align*}
\begin{align*}
    \therefore\quad y&=\frac{0.16}{0.04} \tag{两边同除以0.04}\\
y&=4\text{(张)} \tag{8分邮票张数}\\
12-y&=12-4=8\text{(张)} \tag{4分邮票张数}\\
\end{align*}

【对比】算术解法中,我们要得出关键的算式
(2.1)是不容易的,必须要经过一番思考,化费一定
心血才能得出的.而且每一步都要给出必要的说明,
显得“拐弯抹角”、“道路曲折”;而代数解法从思
路到计算都是比较直截了当,平铺直叙的.再加上我
们只要由例2.1、例2.2两题的代数解法中,不难发现,
它的基本路子、格式是相类似的.也就是说,代数解
法是有普遍性的.特别是在解一些较复杂的应用题
中,这种普遍可行的解法,更能显示出它的优越性
来.

\begin{example}
    有两所图书馆,自建馆以来每年各进图书五
千册;如果今年甲馆藏书23万册,乙馆藏书11万册,
今后仍然是每年各进书5千册.试问:由今年起,几
年后甲馆所藏图书的册数是乙馆的三倍?
\end{example}
    
我们采用直接了当的代数解法,从中进一步明确
这种解法的普遍性、一贯性,并领略它的优越性.

\begin{solution}
首先,设由今年起$x$年后甲馆藏书册数为乙
馆的三倍.

    其次,由于今后每年两图书馆都是仍然进书5千
册,因此,$x$年后:
\begin{itemize}
    \item 甲馆藏书册数为:$(23 + 0. 5x)$万册;
    \item  乙馆藏书册数为:$(11+0.5x)$万册.
\end{itemize}

    再次,由于$x$年后,甲所藏图书的册数是乙的三
倍,所以就有等式:
\[(23 + 0. 5x)=3(11+0.5x) \]

以下只须用“通性”及等式性质进行变形:
\begin{align*}
    23 + 0. 5x& =33+1.5x \tag{分配律}\\
23&=33+1.5x-0.5x  \tag{两边同减$0.5x$}\\
23-33&=1.5x-0. 5x  \tag{两边同减33}\\
23-33&=(1.5-0.5) x  \tag{分配律}\\
-10&=x  \tag{减法法则}\\
x&=-10  \tag{两边调换位置}\\
\end{align*}
    
这就是说:从今年起,$-10$年后(也就是10年前)
甲馆藏书册数就是乙馆的三倍.
\end{solution}

    可见,代数解法因为启用了字母表示所求的数,
所得结果用有理数的意义很容易做出问题的答案,如
果还用算术解法,就要经过一番周折,多费一些心血
去说明了.这也是代数解法较算术解法优越的表现之一.

\begin{ex}
\begin{enumerate}
    \item 试用算术、代数两种解法解下列各题:
    \begin{enumerate}
        \item 某数的4倍减去3,恰等于13,求某数.
        \item 现有一堆小球,分给若干儿童,若每人平均分给5个,
    最后缺2个小球,若每人平均分给4个,又多余3个小
    球.试问:有几个儿童?几个小球?
    \end{enumerate}

    \item 你能说说代数解法有几个步骤吗?
\end{enumerate}
\end{ex}
    
\subsection{未知数和方程式}
  由前边三个应用题的代数解法,不难看出,这种
解法的一般做法要点是:

    用一些字母$x,  y,\ldots$来表示所要求的数量,我
们称为“\textbf{未知数}”.就是用来表达“未知量”的符
号.

    用含有未知数的算式,表达问题中所涉及到的其
它数量.

    把问题中的数量关系,平铺直叙、直截了当地用
等式表示出来,这个等式我们称为\textbf{方程式}.

    运用“通性”与等式性质,由所列方程式求出\textbf{未
知数应有的值}.

    在以上做法中,由引进未知数到列出等量关系式
  (方程),只是解应用题的准备工作;而有效地应用
  “通性”、等式性质却是代数解法的关键所在.但要
注意:准备工作是基础,必须做得充分、熟练、正
确.否则,是谈不上能够正确解决问题的.

    因此,我们还要首先讨论方程式的意义及怎样列
方程式,然后再进一步探讨求出“未知数应有的值”
的原理和方法.

    正如前面三个例题的代数解法中,通过引入未知
数,我们列出了等式:
\[4x+x=138\]
\[0.08y+(12-y)\x 0.04=0.64\]
\[23+0.5x=3(11+0.5x)\]

我们就把这些\textbf{含有未知数的等式,叫做方程式}.
简称\textbf{方程}.又如:$x\cdot y-3x=5$也是方程式.

在一个方程中,所含未知数,又称为\textbf{元};被“$+$”
“$-$”号隔开的每一部分(包括这部分前边的“$+$”、
“$-$”号在内)称为\textbf{一项};在一项中,数字或表示已
知数的字一母因数叫做未知数的\textbf{系数}.并且在各项中,
所含有的未知数的\textbf{次数和},称为\textbf{这一项的次数},如:
$+ 0. 08x$,  $x$, $-1$, $5x$等项的次数都是1,而$xy$项的
次数就是2;不含未知数的项,称为\textbf{常数项},它的次
数是0,因此也称为0次项.

  在一个方程的各项中,最高次项的次数,就称为
这个\textbf{方程的次数}.

  例如:方程$4x+x=138$中,共有三项:$4 x$, $+x$
及138;各项的系数分别是4, $+1$和常数项138;各
项的次数分别是1、1及0;方程的次数是1.

  又如:方程$xy-3x=5$中,共有三项:$xy$, $-3x$
及5;各项的系数分别是$+1$,$-3$及常数项5;各项
的次数分别是2,1及0;方程的次数是2.等
等.

  实际上,方程就是表达已知数与未知数之间的一
种等式关系,这种关系式是解决问题的基础,必须在
明确量与量之间的正确关系的前提下,适当引入未知
数,才能表达出来.

\begin{example}
引入未知数,正确表达出以下问题中的等量
关系:
\begin{enumerate}
    \item 某数的五分之一等于九,求某数.
    \item 某数与它的一半之和,恰好是24,求某数.
    \item 某两班学生总数是101人,而这两班的人
    数相差三人,求这两班各多少人?
\end{enumerate}
\end{example}

\begin{solution}
\begin{enumerate}
    \item 设某数为$x$,则$\frac{x}{5}=9$
    \item 设某数为$y$,则$y+\frac{y}{2}=24$
    \item 设一个班有$x$人,另一班就有
  $(101-x)$人,则  $x-(101-x)=3$ 或   $(101-x)-x=3$.

也可以这样解:    设一个班有$x$人,另一个班就有$(x+3)$人,
      则$$x+(x+3)=101$$
      
还可以引入两个未知数:
    设一个班有$x$人,另一个班有$y$人,
则由题目可知,应该同时有两个关系成立:
\[\begin{cases}
    x+y=101\\
    x-y=3
\end{cases}\]
\end{enumerate}    
\end{solution}


\begin{ex}
\begin{enumerate}
    \item 试说明下列各方程式有哪些项组成?各项的系数是什么?
    方程的次数是几?
    \begin{multicols}{2}
        \begin{enumerate}
 \item $\frac{1}{2}x-5x=7$
 \item $3x^2-5=2x$
 \item $-0.05+7.2x=x$
 \item $x-xy=0$           
        \end{enumerate}
    \end{multicols}

\item 引入一个未知数,列出方程式:
\begin{enumerate}
    \item 用某数的2倍去乘$-\frac{1}{3}$,正好是$-10$,求某数.
    \item 用2元钱去买回若干本书,每本书两角钱,还找回了六角
  钱,问买回几本书?
  \item 高为4米,长比宽多2米的长方体体积为140米$^3$.求这
  个长方体的长和宽.
  \item 两数之和为36,若已知一数是$m$,试求另一数.(注意:
  $m$表示已知数).
  \item 浓度为20\%的糖水300克中,含有多少克糖?多少克水?
  \[\text{浓度}=\frac{\text{糖}}{\text{水+糖}}\]
  \item 浓度为20\%的糖水150克中,加入50克水,浓度就变成
  百分之几?
\end{enumerate}
\end{enumerate}
\end{ex}

\subsection{方程的解与解方程的原理}
    根据应用题所给条件,引入未知数,正确的列出
方程式,只是为解决问题提供了基础,作好了必要的
准备.要使问题得到完满解决,就要从所列方程式出
发,求出未知数应有的值来,并加以验证.

    在这里所说的\textbf{未知数应取的值是指:把所列方程
中的未知数换成这个值以后,就使它变成一个恒等
式}.也就是说:把未知数应取的值代入原方程中,能
使原方程成为“真正等式”.例如:
\begin{itemize}
    \item 把$x = 49$代入$x+ (x+3)=101$,    得$101=101$;
    \item 把$y=7$代入$2-0.2y=0.6$,    得$0 .6=0. 6$;
    \item 把$x=1$与$y=8$同时代入$xy-3x=5$,    得$5=5$,等等.
\end{itemize}
  
象这样,\textbf{能使方程式成为真正等式的未知数的
值,叫做方程的解}.或简单说:\textbf{使方程成立的未知数
的值},叫做\textbf{方程的解}.(一元方程的\textbf{解},也叫做\textbf{根}).例如:
\begin{itemize}
    \item $x = 49$是方程$x+ (x+3)=101$的解(或根);
    \item $y=7$是方程$2-0.2y=0.6$的解(或根);
    \item $x=1$与$y=8$是方程$xy-3x=5$的解(或根)等.
\end{itemize}


\begin{ex}
    观察一下,下列方程的解各是什么?为什么?
    \begin{multicols}{2}
        \begin{enumerate}
 \item $2x+1=6$
 \item $x+3=2x$
 \item $xy=0$
 \item $0.5x+1=3$           
        \end{enumerate}
    \end{multicols}
\end{ex}

求方程的解的过程,叫做\textbf{解方程}.

解方程不能只靠观察、试验,而需要有系统地、
有根据地去探求普遍方法.这就要用“数系运算通
性”及“等式性质”作为我们的有力工具了.

\textbf{解方程的原理}是十分简朴的,那就是:对于一个
方程式中的未知数与已知数进行统一考虑,因为它们
都是数,又同在一个等式当中,所以,它们在运算上
就应该满足“通性”及等式的性质.只要我们有效地
应用这些性质,就可以逐步变形,求出方程式的解
(根)来.

下面以方程$4x+2(12-x) =32$为例,分析一下
是如何有效地应用这些性质(特别是分配律)去求出
方程的解的,从中我们可以总结出代数解法的一般方
法和规律.

首先,由于在方程$4x+2(12-x)= 32$中,含有未
知数$x$,因而无法按运算顺序(先括号里,后括号外)
进行变形.这时候,分配律就要大显身手发挥作用
了,运用它可以“小表及里”地把括号打开,将方程
变形为:$4x+2\cdot 12-2 \cdot  x = 32$;

其次,运用加法交换、结合律及分配律,可以逐
步将方程变形如下:
\begin{equation}
    \begin{split}
        \underline{(4x-2x)}+24&=32\\
        \underline{(4-2)x}+24&=32\\
        \underline{2x}+24&=32
    \end{split}
\end{equation}

由此,可总结出方程变形的第一个规律:

\textbf{运用“通性”}(特别是运用\textbf{分配律})\textbf{可以“由
表及里”的将括号打开,并将“含有相同未知数且所
含未知数的次数也相同”的各项结合起来,合并在一
起}——这叫做\textbf{合并同类项}.从而使方程式简化.

再次,对于方程(2.3),利用等式性质可以继续
变形,将方程$2x+24 = 32$两边同加$(-24)$得:
\[2x = 32 -24\]

可以发现,利用等式性质3进行的这一变形,就
相当于把方程$2x+24 = 32$中的左边一项$+24$,\textbf{改变符
号后,移到右边去},就得到方程式:
\begin{equation}
    \begin{split}
        2x &= 32 -24\\
        2x&=8
    \end{split}
\end{equation}

由此,又可以总结出方程变形的第二个规律:

\textbf{运用等式性质3把方程一边的任意一项改变符号
以后,移到方程的另一边}——这叫做\textbf{移项}.简单说,
就是“移项变号”.

最后,由方程(2.4),只要再用等式性质40就可以变形为最简单的形式,未知数应取的值也就一目
了然了.这就是:

将方程$2x = 8$两边同乘(或同除以)一个非零数$\frac{1}{2}$(或2),即可得出
\begin{equation}
    x=4
\end{equation}

由此,还可以总结出方程变形的第三个规律:

\textbf{运用等式性质4,把方程两边各项同除以未知数
的系数(或乘以系数的倒数)},使方程化为最简形
式.得到未知数应取的值.

综合以上三条规律,得到\textbf{解方程的具体方法是:
展开括号、移项变号、合并同类项、除以未知数的系
数,使方程化为最简形式}.当然,在求出未知数的值
以后,还要代入原方程加以检验,最后确定出原方程
的解.

\begin{example}
    解方程$9x-5=5-x$
\end{example}


\begin{solution}
    先“移项变号”得
                \[9x+x=5+5\]
再“合并同类项”得:
                  \[10x=10\]
最后“除以$x$的系数10”得:
                  \[x=1\]
把$x=1$代入原方程的两边,进行检验:
\[\text{左边}=9\x 1-5=4,\qquad \text{右边}=5-1=4\]

$\therefore\quad $  左边$=$右边

$\therefore\quad x=1$ 是原方程$9x-5=5-x$的解.
\end{solution}


\begin{example}
    解方程$-5=5x-7(1-x)$
\end{example}

\begin{solution}
\begin{align*}
    -5&=5x-7+7x  \tag{用分配律展开括号}\\
  -7x-5x &= 5-7 \tag{移项变号}\\
  -12x&=-2  \tag{合并同类项}\\
  x&=\frac{1}{6}  \tag{两边除以$-12$}
\end{align*}    
把$x=\frac{1}{6}$代入原方程两边,经检验知:
$\frac{1}{6}$是方程 $-5=5x-7(1-x)$ 的解.
\end{solution}


\begin{ex}
\begin{enumerate}
    \item “解方程”和“方程的解”一样吗?区别是什么?
    \item  解方程的原理是什么?根据原理又得出些什么样具体方法
    和规律?
    \item  试用解方程的具体规律,将下列各方程化为最简单形式:
    \begin{multicols}{2}
        \begin{enumerate}
            \item $x-7=9$
            \item $6-y=y-5$
            \item $2x+\frac{1}{2}=0$
            \item $12=6x-9(2-x)$
        \end{enumerate}
    \end{multicols}
\end{enumerate}   
\end{ex}

\section*{习题2.1}
\addcontentsline{toc}{subsection}{习题2.1}

\begin{enumerate}
    \item 试用算术、代数两种解法解下列各题,并比较优劣:
\begin{enumerate}
\item 某生产队今年植树18000棵,正是去年植树的2
倍还多400棵.问:去年植树多少棵?
\item 一本864页的书,每页55行,每行40个字;再版
时计划每页增印5行,每行又多印8个字.问:再版这本书
时能比原版减少几页?
\item 兄、弟二人,今年分别为15岁和9岁.问:几年
后,兄是弟的年龄的2倍?
\end{enumerate}

\item 引入未知数$x$,把下列问题中所求的量,用含有未知数
的算式列出来:
\begin{enumerate}
\item $x$的$1\frac{1}{2}$倍与$-7$的代数和.
\item $x$的相反数与27的差.
\item $x$与已知数$a$的平方和再减去10所得的差.
\item $x$的20\%与51的差的一半.
\item 浓度为51\%的盐水$x$克中,含有的纯盐量与含水
量.
\item $x$克盐溶化到100克水中,盐水的浓度.
\item 一件工作,小李$x$天作完,求:每天的工作量和
3天的工作量.
\item 以每小时20公里的速度,要走出$x$公里的路程,
需要多少时间?
\item 操场上有400米一圈的跑道,速度分别是$x$米/秒
和2$x$米/秒的两个人,同时,同地相背而跑时,在什么时间
相遇?
\item 上题中,如果两人同向跑时,快的在什么时间追
上慢的.
\end{enumerate}

\item  适当引入未知数,列出下边问题的方程:
\begin{enumerate}
\item 某数与五的和,正好是这个数的3倍,试求某
数.
\item 矩形的周长是40,长比宽多10,试求这个矩形的
面积.
\item 浓度为85\%的酒精100斤与浓度为15\%的酒精100
斤混合在一起以后,它的浓度是多少?
\item 甲、乙二人由同地出发,甲以每小时走5里的速
度先出发1.5小时,乙骑自行车经过50分钟才赶上甲.试
求:乙每小时骑车能走多少里?
\end{enumerate}

\item 指出下列各方程中:有那些项,各项的系数、次数分别
是多少?方程是几次的?
\begin{multicols}{2}
\begin{enumerate}
    \item $7x+1=0$
    \item $-\frac{1}{5}x-1=x+10$
    \item $0=0.1x$
    \item $1-xy=3x$
\end{enumerate}
\end{multicols}

\item  利用解方程原理得出的具体规律,把下列方程式化为
最简形式:
\begin{multicols}{2}
\begin{enumerate}
    \item $\frac{1}{2}x=-1$
    \item $\frac{1}{3}x+1=\frac{1}{3}$
    \item $3-x=6$
    \item $-7+2(1-x)=0$
\end{enumerate}
\end{multicols}

\item  解下列方程,并注明每步变形的根据:
\begin{multicols}{2}
    \begin{enumerate}
        \item $x-1=2x$
        \item $0=7x-\frac{1}{2}$
        \item $8-10y=-32-5y$
        \item $\frac{1}{4}=2x-\frac{1}{2}(1-4x)$
    \end{enumerate}
    \end{multicols}

\item  如果$a,  b$都是已知的数,并且$a\ne -1$,试利用解方程的
原理,解方程:
\[ax+1=b-x,\qquad b-ax=\frac{1}{2}+x   \]

\end{enumerate}

\section{一元一次方程}
\subsection{一元一次方程}
    在第一节所列举的方程式中,可以找出具有这样
特点的方程:
\begin{enumerate}
    \item 只含有一个未知数;
    \item 分母不含有未知数;
    \item 方程的次数是1次.
\end{enumerate}
比如:$4x+x = 138$;$\frac{x}{5}=9$;$0.08y+0.04(12-y)=0.64$;$x+\frac{x}{2}=24$等
都属于这一类方程式.

    我们把\textbf{只含一个未知数、分母不含未知数,且次
数又是1的方程,称为一元一次方程}.

一元一次方程的一般形式是:
\[ax+b=0\quad \text{($a,b$是已知数,且$a\ne 0$)}\]

\subsection{一元一次方程的解法}
    我们利用由“通性”及等式性质而归纳出来的代
数解法的具体规律,来讨论一元一次方程的解法.

\begin{example}
    解方程$8x=5x-3$
\end{example}


\begin{solution}
    原方程移项得\quad $8x-5x=-3$

    合并同类项得\quad $3x=-3$

    两边除以3得\quad $x=-1$
\end{solution}

\textbf{验算:} 把$x=-1$代入原方程中的两边,
\[\begin{split}
    \text{左}&=8\x(-1)=-8,\\
    \text{右}&=5\x(-1)-3=-8,
\end{split}\]
左边$=$右边,$\therefore\quad $原方程的解是$x=-1$.

\begin{example}
    解方程$7(1-x)=2(x+3)-4 (5+4x)$.
\end{example}


\begin{solution}
    先运用分配律去括号:
    \[7-7x=2x+6-20-16x\]
移项:把含$x$的各项移至方程左边,而把所有的常数项移至右边.
   \[ -7x-2x+16x=6-20-7\]
合并同类项:分别求含x项的系数的代数和与常数项的代数和:
    \[7x=-21\]
两边除以7:$x=-3$.
\end{solution}

\textbf{验算:}
$x=-3$代入原方程两边,
\[\begin{split}
    \text{左}&=7[1-(-3)]=28,\\
    \text{右}&=2(-3+3)-4 [5+4(-3)]=28,
\end{split}\]
左边$=$右边,$\therefore\quad $原方程的解是$x=-3$.



\begin{example}
    解方程$3(3y+1)=2(1+y)+3(y+3)$
\end{example}

\begin{solution}
    把原方程去括号: $9y+3=2+2y+3y+9$

    移项:$9y-2y-3y=2+9-3$

    合并同类项:$4y=8$
       
    除以$y$的系数4:$y=2$
\end{solution}

同学可以自己验算,从而得到: 原方程的解是$y=2$.


\begin{ex}
解下列方程: 
\begin{multicols}{2}
    \begin{enumerate}
\item $0=5x-1$
\item   $x+5=-2-6x$
\item $3(1-x)+1=-(x-1)$      
\item $5(x-1)-3(x-1)=0$
    \end{enumerate}
\end{multicols}
\end{ex}

\begin{example}
    解方程$0.3(8x-1)=2x+2.9$
\end{example}

\begin{solution}
\begin{align*}
    3 (8x-1)&=20x+29  \tag{原方程两边乘以10}\\
    24x-3&=20x+29   \tag{去括号}\\
    24x-20x&=29+3\tag{移项变号}\\
    4x &= 32\tag{合并同类项}\\
    x&=8\tag{两边除以4}\\
\end{align*}
经验算以后,可知:
原方程的解是$x=8$.    
\end{solution}

\begin{ex}
解下列方程:
\begin{multicols}{2}
\begin{enumerate}
    \item $0.1x+1=0.2$
    \item $0.16(x+5)=1-0.2x$
    \item $0.625=0.5(y-2)$
    \item $-0.1(x-10)=9-0.2(1-x)$
\end{enumerate}
\end{multicols}
\end{ex}

\begin{example}
  解方程
$\frac{2y-1}{3}-1=\frac{5y+1}{8}-\frac{3y+1}{6}$
\end{example}

\begin{analyze}
   遇到分数系数的方程时,可以先由等式
    性质,两边乘以“各分母的最小公倍数”,把系数的
    分母去掉,转化为整数系数方程以后,再求解.本例
    中3,8,6的最小公倍数为24.
\end{analyze}

\begin{solution}
    方程两边各项乘以24
 \begin{align*}
    8 (2y-1)-24&=3 (5y+1)-4 (3y+1)  \tag{去分母}\\
    16y-8-24&=15y + 3-12y-4  \tag{去括号}\\
    16y-15y+12y&=3-4+8+24  \tag{移项}\\
    13y&=31    \tag{合并同类项}\\
    y&=\frac{31}{13}=2\frac{5}{13} \tag{两边除以13}
 \end{align*}   
 经验算(可以口算或在草稿纸上笔算,不写入解
 题过程中)后知道,原方程的解是$y=2\frac{5}{13}$.
\end{solution}

\begin{ex}
解下列方程:
\begin{multicols}{2}
\begin{enumerate}
    \item $\frac{1}{5}+x=-1$
    \item $\frac{1}{4}(1-5x)=0.5$
    \item $\frac{3}{7}(1-x)=1-\frac{x}{7}$
    \item $-\frac{1}{8}(24-16x)=-\frac{1}{9}(18x+9)$
\end{enumerate}
\end{multicols}
\end{ex}

归纳以上例题,可以看出:
\begin{blk}{}
    解一元一次方程的一般步骤为
    \begin{enumerate}[I.]
        \item 去分母(或化为整系数).
        \item 去括号.
        \item 移项变号.
        \item 合并同类项、化为$ax=-b$的形式$(a\ne 0)$.
        \item 除以未知数的系数得:$x=-\frac{b}{a}$.
    \end{enumerate}
\end{blk}
      
这里还要指出:
\begin{enumerate}
    \item 由于方程的形式多样,在解的
过程中不必死套以上五步,要根据具体情况,灵活应
用.
\item 验算可以不必写出,但一定要自己口算或笔算
进行检验,以确实保证计算正确.
\end{enumerate}

\begin{example}
解下列方程:
\begin{enumerate}
    \item $25(1-2x)=12.5+12.5(1-4x)$
    \item $\frac{1}{2}(4x+6)=-1+2x$
\end{enumerate}
\end{example}

\begin{solution}
\begin{enumerate}
    \item     观察所给方程,显然用不着化小数系
    数为整数,去括号以后,一合并即可“凑整”.因
    此
    \begin{align*}
25-50x&=12.5+12.5-50x  \tag{去括号}\\
    -50x+50x&=25-25  \tag{移项}\\
    0\cdot x&=0 \tag{合并同类项}    
    \end{align*}
    
    显然,$x$可以是任意数.

    事实上,当$x$取任意数时,原方程两边都能得
到相等的结果.

    所以,原方程的解是:$x$取任意数(无限多个).
    
\item     \begin{align*}
    2x+3&=-1+2x  \tag{去括号}\\
        2x-2x&=-1-3  \tag{移项}\\
        0\cdot x&=-4 \tag{合并同类项}    
        \end{align*}
        显然,无论$x$取什么值,方程的左边总是0,
    边总是$-4$,不可能使左、右相等;这也就说明,
    论x取什么值,总是不能使原方程式两边的值相等.

        这时,我们就说:原方程式无解.
\end{enumerate}    
\end{solution}

\begin{ex}
解下列方程:
\begin{enumerate}
    \item $4\left[\frac{1}{4}-\frac{1}{2}(3x-1)\right]=2[0.5(-3-x)+1]$
    \item $7-9x=3\left(2\frac{1}{3}-x\right)-6x$
    \item $\frac{1}{5}(10-20x)=7-4x$
    \item $\frac{x}{2}+\frac{x}{3}+\frac{x}{6}-7=2x+5$
\end{enumerate}
\end{ex}


\begin{example}
    解方程$\frac{1}{2}\left\{\frac{1}{3}\left[\frac{1}{4}\left(\frac{0.1x-0.5}{0.5}\right)-2\right]-3\right\}-4=0$
\end{example}

\begin{analyze}
方程左边的算式很繁,但很有规律,特
    别是小括号内的算式$\frac{0.1x-0.5}{0.5}$,可以利用分数的
    基本性质,分子、分母同乘以10,就变成为
$\frac{x-5}{5}=\frac{1}{5}x-1$,
    因此原方程就可以写成:
   \[\frac{1}{2}\left\{\frac{1}{3}\left[\frac{1}{4}\left(\frac{1}{5}x-1\right)-2\right]-3\right\}-4=0 \]
    这时,也不一定非要先去括号,或先去分母,那样计
    算稍繁,为计算方便,可逐步进行去分母与合并.
\end{analyze}

\begin{solution}
    原方程式可写成:\[\frac{1}{2}\left\{\frac{1}{3}\left[\frac{1}{4}\left(\frac{1}{5}x-1\right)-2\right]-3\right\}-4=0 \]
\begin{align*}
    \frac{1}{3}\left[\frac{1}{4}\left(\frac{1}{5}x-1\right)-2\right]-3 &=8  \tag{两边乘以2}\\
    \frac{1}{3}\left[\frac{1}{4}\left(\frac{1}{5}x-1\right)-2\right]&=11   \tag{移项、合并}\\
    \frac{1}{4}\left(\frac{1}{5}x-1\right)-2&=33  \tag{两边乘以3}\\
    \frac{1}{4}\left(\frac{1}{5}x-1\right)&=35   \tag{移项、合并}\\
    \frac{1}{5}x-1&=140 \tag{两边乘以4}\\
    \frac{1}{5}x&=141  \tag{移项、合并}\\
    x&=705  \tag{两边乘以5}\\
\end{align*}   

经检验,原方程的解是$x=705$.
\end{solution}

\begin{ex}
解下列方程:
\begin{enumerate}
    \item $\frac{0.1-0.3x}{0.2}=5$
    \item $\frac{x-3}{0.5}-\frac{x+4}{0.2}=16$
    \item $\frac{3}{2}\left[\frac{2}{3}(x-1)-2\right]=3$
    \item $\frac{1}{3}\left\{\frac{1}{3}\left[\frac{1}{3}\left(\frac{1}{3}-1\right)-1\right]-1\right\}-1=0$
\end{enumerate}
\end{ex}

\section*{习题2.2}
\addcontentsline{toc}{subsection}{习题2.2}

\begin{enumerate}
\item 检验以下各题方括号中所给出的数是不是所给方程的解
(或根)?
\begin{enumerate}
    \item $2 (3x-4)=5(x-2),\qquad [3,\; -2]$
    \item $ x(x+ 1)=12,\qquad [3,\; -3,\; 4,\; -4]$
    \item $\frac{x}{2}+\frac{x}{3}+\frac{x}{4}=x,\qquad \left[-0.1,\; -2,\; \frac{1}{4},\; 7,\; -101,\; 0,\; 1981\right]$
    \item $9-10x=\frac{1}{4}(2-36x)-x,\qquad [0,\; 1,\; -1,\; a]$
\end{enumerate}

\item 解方程:
\begin{multicols}{2}
\begin{enumerate}
    \item $0.8x=1$
    \item $7x-1=5x$
    \item $0=6x-\frac{1}{2}$
    \item $4-7x=7x-4$
    \item $2|x|=2$
    \item $1=3-|x|$
\end{enumerate}
\end{multicols}

\item 解方程:
\begin{enumerate}
    \item $-5(x-1)=0$
    \item $1-2(x-3)=x$
    \item $10y+7=3(4y-1)+4(1-y)$
    \item $9=2(3x-1)+x-4$
    \item $4(2t+3)=8(1-t)-5(t-2)$
    \item $5(z-1)-9(1-z)+3(2z-2)-(z-1)=2$
\end{enumerate}

\item 解下列方程:
\begin{enumerate}
    \item $-0.3(1-x)+0.1=2$
    \item $0.7=x-0.2(5-x)$
    \item $0.2(2x-1)-0.5(2-4x)+1=0$
    \item $7(2x+1)-3(4x+2)+5(x+0.5)-1=0$
    \item $1.8-8x-0.6(1.3-3x)=4(5x-0.4)$
\end{enumerate}

\item 解方程:
\begin{multicols}{2}
\begin{enumerate}
\item $\frac{7x-5}{4}=\frac{3}{8}$
\item $\frac{3-x}{2}=\frac{x-4}{3}$
\item $\frac{2x-1}{6}-\frac{5x-1}{8}=1$
\item $\frac{3x-1}{2}+x=\frac{2x+1}{5}-1$
\item $y-\frac{y-1}{2}=2-\frac{y-2}{5}$
\item $\frac{2}{5}y+\frac{1}{9}=\frac{1}{9}y-\frac{2}{5}$
\item $\frac{x-2}{5}-\frac{x+3}{10}-\frac{2x-5}{3}+3=0$
\item $\frac{9t+2}{7}-\frac{3+2t}{3}-\frac{3t-14}{2}=1$
\item $1\frac{1}{2}x-\frac{14-x}{3}=3x$
\item $x-\frac{3x+6}{2}+\frac{3x+6}{5}=0.5x$    
\end{enumerate}
\end{multicols}
\item 用解方程的方法,求以下问题中的未知量:
\begin{enumerate}
    \item 已知公式$\ell=\ell_0(1+\alpha t)$中,$\ell= 80.096$,$\ell_0 = 80$,$\alpha=0.000012$,试求$t$.
\item 已知一梯形的面积是120$cm^2$,上底是12$cm$,高
是8$cm$,试求下底长是多少?
\item 在公式$S=\frac{1}{2}at^2$中,$S=168$, $t=4$,试求$a$.
\item 在公式$F=32+1\frac{4}{5}C$中,已知$F=77$,试求$C$.
\end{enumerate}

\item 下列各方程的解法是否正确?如果有错,请把它改正过
来:
\begin{enumerate}
    \item 解: 
    \[\begin{split}
        3(x-1)&=7-x\\
        3x-3&=7-x=3x+x=7+3\\
        &=4x=10\\
        &=x=2\frac{1}{2}
    \end{split}\]
    $\therefore\quad x=2\frac{1}{2}$
    \item 解: 
    \[\begin{split}
        \frac{y}{2}+\frac{y}{4}&=1\\
        2y+y&=1\\
        3y&=1
    \end{split}\]
    $\therefore\quad y=\frac{1}{3}$
    \item 解: 
    \[\begin{split}
        \frac{x+1}{3}-\frac{x-2}{6}&=\frac{4+x}{2}\\
        2x+2-x-2&=12+3x\\
        2x-x-3x&=12\\
        -2x&=12
    \end{split}\]
    $\therefore\quad x=-6$
    \item 解:
    \[\begin{split}
        \frac{0.1x-0.5}{0.2}+1&=0\\
        \frac{x-5}{2}+10&=0\\
        x-5+20&=0\\
        x&=-20+5
    \end{split}\]
    $\therefore\quad x=-15$
\end{enumerate}

\item 解下列方程:
\begin{enumerate}
\item $\frac{0.4 y+0.9}{0.5}-\frac{0.03-0.02 y}{0.03}-\frac{y-5}{2}=0$
\item $x-\frac{1}{2}\left[x-\frac{1}{2}(x-1)\right]=\frac{2}{3}(x-1)$
\item $\frac{1}{2}\left\{\frac{1}{2}\left[\frac{1}{2}\left(\frac{1}{2} y-3\right)-3\right]-3\right\}-3=0$
\item $\frac{1}{8}\left\{4\left[\frac{5}{8}(t-1)+\frac{3}{8}(1-t)\right]-7(1-t)\right\}=100$
\item $\frac{9}{8}\left[\frac{2}{3}\left(\frac{4}{5} x-\frac{2}{3}\right)+1\right]=\frac{2}{3} x\left(\frac{1}{2}-\frac{1}{3}\right)-\frac{5}{8}$
\item $\frac{1}{2}\left\{x-\frac{1}{3}\left[x-\frac{1}{4}\left(x-\frac{2}{3}\right)-\frac{3}{2}\right]\right\}=x+\frac{3}{4}$
\end{enumerate}

\item 解下列方程:
\begin{multicols}{2}
\begin{enumerate}
    \item $|x-4|=0$
    \item $|x+1|=5$
    \item $|x-1|+1=2$
    \item $|2-x|=3|x-2|$
\end{enumerate}
\end{multicols}

\item 解下列文字系数的一元一次方程:
\begin{multicols}{2}
    \begin{enumerate}
        \item $ax+1=0\quad (a\ne 0)$
        \item $5x-a+1=0$
        \item $ax+5=x-3\quad (a\ne 1)$
        \item $ax+b=8-x\quad (a\ne -1)$
        \item $\frac{x}{a}+\frac{x}{b}=1\quad (a+b\ne 0)$
    \end{enumerate}
    (这里的字母$a,b$都是已知数)
    \end{multicols}

    \item 如果已知方程$mx+2=2(m-x)$的根是$x=\frac{1}{2}$,试
求这个方程中$m$的值.
\end{enumerate}


\section{一次方程组}
\subsection{二元一次方程}
    先看下边的实例:

    今有货物10吨,要用大、小两种卡车一次运走,
如果每辆大车能装2吨货,每辆小车只能装1吨货.
试问:应派大、小车各几辆才能在每辆车都装满的情
况下,正好一次运走全部货物?
    
显然,派车的方案不止一种,比如:派五辆大车
而不派小车;或派四辆大车和两辆小车等等,一共有
多少种可行方案呢?我们用代数解法去讨论.

设派出大车$x$辆,小车$y$辆,则由问题所给出的条件可以得出:
\begin{equation}
    2x+y=10
\end{equation}

显然这是一个方程式,它与一元一次方程相比,
只是多了一个未知数(元).而其余的特点是一样
的:分母不含未知数、方程的次数是1,因此,我们
就把“\textbf{含有两个未知数且分母不含未知数的一次方
程}”,叫做\textbf{二元一次方程式}.

\textbf{能够使二元一次方程两边的值相等的未知数$x,y$
的一组值},叫做这个\textbf{二元一次方程的一个解},记作
$(x,y)$.比如:方程(2.6)的一个解,可以记作:
 $ (x,   y)=(5,  0)$,这就是说,把$x=5$, $y=0$这一
组值代入(2.6)的两边,计算结果是相等的.

    前边已经说过,我们所提问题的“派车方案”不
止一种,因此,方程(2.6)的解就不止一个.但由于
这里的$x,  y$都表示所派的卡车辆数,因而它们的值
就只能是\textbf{非负整数}了.否则,所得结果就毫无意义
了.用列表的方法,很容易把所有的“派车方案”
  (也就是方程(2.6)的解)找出来,这就是:
\begin{center}
    \begin{tabular}{c|cccccc}
        \hline
  $x$ &5&4&3&2&1&0\\
  \hline    
  $y$ &0&2&4&6&8&10\\
  \hline
    \end{tabular}
\end{center}
其中每一个$(x,y)$的\textbf{数组值},就是方程(2.6)的一
个解,也就是所提问题的一种“派车方案”.

    把所有能使方程(2.6)成立的,且符合实际问题
的各个\textbf{数组值}所组成的\textbf{集合},可以表示为:
\[A=\{(x,y)\}=\{(5,0),\; (4,2),\;(3,4),\;(2,6),\;(1,8),\;(0,10)\}\]

这个集合中的每一组值,都是(2.6)的一个解,
我们把集合刀叫做方程(2.6)的符合题意的\textbf{解集}.

    应该指出:如果不考虑具体问题,只是从数学上
考虑方程$2x + y=10$的解时,就不能限制在\textbf{非负整数}
的范围内了,而应该考虑到所有的有理数范围.这
时,只要任意取$x$的一个有理数值,代入方程(2.6)
就可以求出$y$的一个相应的值,从而得到方程(2.6)
的一个解;当然,也可以先随便取$y$的一个值,代入
方程(2.6),就能求出$x$的一个相应值,同样可得到
方程(2.6)的一个解.这样可以无止境的作下去,因
此,可以得到方程(2.6)的无限多个解.如下表:
\begin{center}
    \begin{tabular}{c|ccccccccc}
        \hline
  $x$ &$\cdots$& $-3$ & $-\frac{1}{2}$  & $-1$  & $0$  & $\frac{1}{2}$ & $0.7$ &$10.6$  &$\cdots$\\
  \hline    
  $y$ &$\cdots$& $16$ & $11$  & $12$  &  $10$ & $9$ & $8.6$  &$-11.2$ &$\cdots$\\
  \hline
    \end{tabular}
\end{center}

所以,我们说:
\textbf{任一个二元一次方程有无数多个解}.正因为如
此,二元一次方程也被称为\textbf{不定方程}.

\begin{example}
    试求二元一次方程$3x +y=8$的正整数解.
\end{example}

\begin{analyze}
    二元一次方程虽有无数多个解,但本题
只要求\textbf{正整数解},因而\textbf{可能}是只有有限几个解.
\end{analyze}

\begin{solution}
可将原方程变形为:
\begin{align*}
    y=8-3x   \tag{移项变号}
\end{align*}

  然后,可以设$x$分别取正整数1、2,相应地代
入上式,求出$y$值分别为:
\[y=8-3\x1=5, \qquad  y=8-3\x 2=2\]

    如果继续取$x=3$,代入上式可知,$y=8-3\x3=-1$,这已经不符合所要求的“正整数”解.显然,
$x$取比3大的正整数时,相应的$y$值更不是“正整
数”了,所以方程$3x +y=8$的正整数解集是:
\[\{(x,y)\}=\{(1,5),\; (2,2)\}  \]

\end{solution}

\begin{ex}
\begin{enumerate}
    \item 求$3y=9-6x$的非负整数解集.
    \item 求$3x+2y=16$的正整数解集.
\end{enumerate}
\end{ex}
          
\subsection{方程组与方程组的解}
    如果我们把前边所提到的实例,另加要求条件,
改为“要用大、小卡车共六辆,一次运走10吨货物,
大车每辆装满2吨,小车每辆装满1吨.试问:应如
何派车?

    不难发现,只是增加了一个条件“只能派六辆
车”.也就是说,用代数法解决这个问题时,不仅可
列出方程
\begin{equation}
    2x+y=10
\end{equation}
而且还应该增加一个条件:
\begin{equation}
    x+y=6
\end{equation}

(2.8)式同样是一个二元一次方程,其中的$x,y$同样表示大、小卡车的辆数,也应限制在\textbf{非负整数}
范围内取值.它的解也列表如下:
\begin{center}
    \begin{tabular}{c|ccccccc}
        \hline
$x$ &6&5&4&3&2&1&0\\
\hline
$y$ &0&1&2&3&4&5&6 \\
\hline      
    \end{tabular}
\end{center}

因此,方程(2.8)的非负整数解集为:
\[B=\{(x,y)\} =\{(6,0),\; (5,  1),\; 
      (4, 2),\; (3,  3),\; (2,   4),\; 
      (1,5),\; (0,6)\} \]

要解决所提的问题,只要求出同时适合方程(2.7)
和方程(2.8)的数组值,也就是求出非负整数解集$A$
与解集$B$的\textbf{公共解解集}就可以了.这样的解集中的数
组值,一定满足方程(2.7)与方程(2.8),因而也就
是我们所提问题的解决方案.

显然,这个公共解就是:
          \[(x,y)=(4, 2)\]
也就是说,只要“派出大车4辆,小车2辆”就符合
这个问题的要求.

    在此,我们把\textbf{集合$A$与集合$B$的公共部分所组成
的新集合},叫做集合$A$与$B$的\textbf{交集合},简称\textbf{交集},记作:$A\cap B$,读成“$A$与$B$的交”.

    如上例:
\[\begin{split}
    A&=\{(5,  0),\; (4, 2),\; (3,  4),\;  (2,  6),\; (1,  8),\; (0,  10)\} \\
B&=\{(6,  0),\; (5,  1),\; (4,  2),\; (3,  3),\; (2,  4),\; (1, 5),\;     (0,  6)\}\\
A\cap B&=\{(4,  2)\}
\end{split}  \]

\begin{figure}[htp]
\begin{center}
    \begin{tikzpicture}[scale=1.5]
\draw (0,0) circle (1.2);
\draw (2,0) circle (1.8);

\fill [pattern=north west lines] (0.55,1.07) to [bend left=28] (1.2,0) to [bend left=28]  (0.55,-1.07)to [bend left=34]  (0.55, 1.07);

\node at (0,.8){$A$};
\node at (.7,.25)[fill=white]{$A\cap B$};
\node at (.7,-.25)[fill=white]{$(4,  2)$};
\node at (2.2,1.5){$B$};

\node at (-.4,.5){$(5,0)$};
\node at (-.4,0){$(3,  4)\;  (2,  6)$};
\node at (-.4,-.5){$(1,  8)\; (0,  10)$};

\node at (2.5,.6){$(6,  0)\quad (5,  1)$};
\node at (2.5,0){$(3,  3)\quad (2,4)$};
\node at (2.5,-.6){$(1, 5)\quad  (0,  6)$};
    \end{tikzpicture}
\end{center}
    \caption{}
\end{figure}

由此可见,在解决应用问题时,还可以引入两个
或更多个未知数,列出几个方程式,把它们联合起来
求得公共解.

    我们把\textbf{几个方程式联合在一起,组成一个整体},
就叫做\textbf{联立方程式},也叫\textbf{方程组}.一个方程组中的几个方程,应用“\{”标出.如:
\[\begin{cases}
   2x+y=10\\
   x-y=6 
\end{cases},\qquad \begin{cases}
    2x=y-5\\
    3x=-4y
\end{cases}\]    
都表示方程组.

\textbf{含有两个未知数的一次方程组,称为二元一次方
程组}.比如:
\[\begin{cases}
    x+y=2\\
    y=1 
 \end{cases},\qquad \begin{cases}
     x-2y=3\\
     x+y=7
 \end{cases},\qquad \begin{cases}
    x=2+y\\
    y=-x
\end{cases}\]   
都是二元一次方程组.

\textbf{能够同时满足方程组中每一个方程的未知数的一
组值},叫做\textbf{方程组的解}.

\textbf{也就是说:一个方程组中,每个方程的解集的交
集},就是这一\textbf{方程组的解集}.例如:
    $x=4$,  $y=2$能够同时满足$2x+y=10$与$x+y=6$,
因此,$(x,y)=(4,2)$就是方程组
$\begin{cases}
    2x+y=10\\
    x+y=6 
 \end{cases}$的一个解.

\begin{ex}
\begin{enumerate}
    \item 检验下列方括号中的数值组$(x,y)$,
    有没有方程组$\begin{cases}
        2x+y=-5\\
        3x=-4y 
     \end{cases}$的解?
    \[[(4,3),\quad (4,-3),\quad (-4,3), \quad (0,-5)  ] \]

    \item 如果给你一个数值组$(x,  y) = (1,1)$.你能造出一个二元
    一次方程组,使它的解是这个数值组吗?
\end{enumerate}
\end{ex}

\subsection{二元一次方程组的解法}
    求方程组的解的过程,叫做\textbf{解方程组}.

    如何解二元一次方程呢?

    假如按照上边所说方法:“先求出方程组中每一
个方程的解集,再求这些解集的\textbf{交集}”去作的话,就
会发现:这种方法按理说没有问题,但实际作起来既
繁,又没有准儿.因为每一个方程都有无限多个解,
全部找出来不易;要求每个方程的公共解,就更不容
易了.为此,我们需要进一步探求二元一次方程组求
解的普遍可行的有效方法.

    一般来说,解二元一次方程组的关键是:\textbf{设法把
二元转化为一元方程求解}.总称为\textbf{消元法}.

\subsubsection{代入消元法}



\begin{example}
解方程组
    \begin{numcases}{}
      2x+y=13\\
      7x+9y=84  
    \end{numcases}
\end{example}

\begin{analyze}
方程(2.9), (2.10)既然组成一个方程
组,因而两个方程式中的未知数$x$就是表示同一个数
量,而未知数$y$也表示另一个相同的数量.这样,就
可以从(2.9)中首先解出:$y= 13-2x$ (移项变号),
再把这个关系式代入(2.10)式中,去代换(2.10)中的$y$.
从而使(2.10)式中消去未知数$y$,变成:
              \[7x+9(13-2x) = 84\]
解这个一元一次方程,就能求出二值.再利用(2.9)式,
就能求出$y$的相应值.
\end{analyze}

\begin{solution}
由(2.9)式可得
\begin{equation}
    y=13-2x
\end{equation}
把(2.11)式代入(2.10)式,可得
    \[7x+9(1.3-2x)=84\]
解这个一元一次方程,经过去括号、移项变号及
合并同类项,可得
        \[-11x=-33\]  
$\therefore\quad x=3$

把$x=3$再代入(2.11)式,可得
\[y=13-2\x3 =7\]
\end{solution}

\textbf{验算:} 把$x=3$, $y=7$同时代入(2.9)与(2.10)的两
边,代入(2.9):
\[ \text{左}=2\x 3+7=13,\qquad \text{右}=13\]
\[\text{左边}=\text{右边}\]

代入(2.10):
\[\text{左}=7 \x 3+9 \x7=84,\qquad \text{右}=84\]
\[\text{左边}=\text{右边}\]

因此,原方程组的解是$(x,y)=(3,7)$.
即,原方程组的解集为:$\{(x,y)\}=\{(3,7)\}$.

\begin{example}
    解方程组
\begin{numcases}{}
    3x+4y=6\\
    2x+3y=5
\end{numcases}
\end{example}

\begin{solution}
由(2.13)式得:
\begin{equation}
    x=\frac{5-3y}{2}
\end{equation}
(2.14)代入(2.12)得:
\[3\x \frac{5-3y}{2}+4y=6 \]
解这个一元一次方程
\[\begin{split}
    15-9y+8y&=12\\
    -y&=-3\\
    y&=3
\end{split}\]
把$y=3$代入(2.14)得
\[x=\frac{5-3\x 3}{2}=-2\]
经检验可知,原方程组的解是$(x,y)=(-2,3)$,即原方程组的解集是
\[\{(x,y)\}=\{(-2,3)\} \]
\end{solution}

通过以上两例,可以总结这种解法的步骤和要点如下:
\begin{blk}{}
\begin{enumerate}[I. ]
    \item 由方程组中的任一个方程出发,把一个
    未知数写成含有另一个未知数的算式.
    \item 把这个算式代入另一个方程中去,使它
    转化为一元方程式,达到\textbf{消元}的目的.
    \item 解所得的一元方程.
    \item 得到一个未知数的值以后,把它再代回
    I所得的算式(其实,代入原方程组中的任一
    个方程中都可以),进而求得相应的另一个未知
    数的值.
\end{enumerate}    
\end{blk}

这种求解的方法,叫做\textbf{代入消元法},简称\textbf{代入法}.
\begin{ex}
    用代入法解下列方程组:
\begin{multicols}{2}
\begin{enumerate}
    \item $\begin{cases}
        2x+3y=40\\ y=2x
    \end{cases}$
    \item $\begin{cases}
        7x+2y=11\\ x-y=-1
    \end{cases}$
    \item $\begin{cases}
        5x=7y-1\\ 5x=17-14y
    \end{cases}$
    \item $\begin{cases}
        x+y=1\\ 10x+y=5\frac{1}{2}
    \end{cases}$
    \item $\begin{cases}
        \frac{x}{2}+\frac{y}{2}=\frac{1}{2}\\ 2x-4y=-1
    \end{cases}$
    \item $\begin{cases}
        3x+5y-8=0\\ 7x-4y-3=0
    \end{cases}$
\end{enumerate}
\end{multicols}
\end{ex}

\subsubsection{加减消元法}
\begin{example}
    解方程组
    \begin{numcases}{}
        x+y=12\\
        3x-y=4
    \end{numcases}
\end{example}

\begin{analyze}
由于每一个方程都是一个含有未知数
的等式,而我们所要求的正是使每个方程能同时成为
真正等式时未知数的值.因而,等式所具有的性质,
对这些方程当然应该有效,特别是:两个等式的两边
分别相加(或相减),其结果仍是一个等式,即如果
$A=B$,$C=D$,那么就有$A\pm C=B\pm D$.对方程也
应适用.

    根据这个道理,再观察这个方程组中的两个方程
式,发现它们当中“同一个未知数$y$的系数是互为相
反数”.因而,可以把两个方程式的两边分别相加,
消去未知数$y$,得到一个只含有$x$的一元方程,从而
就能解出$x$值,再进一步利用原方程组中的任一个方
程,求出相应的$y$值.使问题得到解决.
\end{analyze}

\begin{solution}
将方程(2.15),(2.16)相加,(2.15)+(2.16)得:
\[4x=16  \qquad \therefore\quad x=4 \]
把$x=4$代入(2.15):
\[4+y=12 \qquad \therefore\quad y=8\]
经检验,原方程组的解是$(x,y)=(4,8)$,即:原方程组的解集是
\[\{(x,y)\}=\{4,8\}\]
\end{solution}

通过这个例题的分析与解法,试想一想:在什么情况下,两个方程相加能消元?又在什么情况下,两个方程相减能消元?

不难知道,只有在\textbf{两个方程式中,同一个末知数
的两个系数绝对值相等}的情况下,才能\textbf{相加}或\textbf{相减达
到消元}.而且,当某一未知数的\textbf{系数相同}时,可以\textbf{相
减};当某一未知数的\textbf{系数互为相反数时},可以\textbf{相加}.


\begin{example}
    解方程组
\begin{numcases}{}
    2x+3y=8\\
    3x+4y=11
\end{numcases}
\end{example}

\begin{analyze}
这个方程组中,未知数的系数没有什
么特点,不具备“相加或相减”进行消元的条件.但
是,我们可以应用等式的性质,设法创造条件,也就
是设法使其中某一个未知数的两个系数绝对值相等以
后,再去消元.

    根据这个想法,就要用等式性质:

如果$A=B$, $C=D$,且有非零数$m, n$.那么,就
有等式:$mA+nC= mB + nD$
\end{analyze}

\begin{solution}
方程(2.17)两边同乘以3.(简写为:(2.17)$\x 3$),得:
\begin{equation}
    6x+9y=24
\end{equation}
(2.18)$\x 2$得:
\begin{equation}
    6x+8y=22
\end{equation}
(2.19)$-$(2.20)得:$y=2$

把$y=2$代入(2.17)(或代入(2.18))得
\[2x+3\x 2=8\qquad \therefore\quad x=1\]
经检验,原方程组的解是$(x,y)=(1,2)$,即:原方程组的解集是
\[\{(x,y)\}=\{(1,2)\}  \]
\end{solution}

通过例2.17,例2.18的分析与解法,可以总结出这种
解法的步骤与要点如下:
\begin{blk}{}
\begin{enumerate}[I. ]
    \item   两方程中,若有“\textbf{同一未知数的两个系
    数绝对值相同}”的特点,可以\textbf{相加或相减}进行\textbf{消
    元};若没有上述特点,可以运用等式性质,先使
    原方程组变形为具有这一特点的形式,然后再把
    变形后的两方程\textbf{相加或相减},达到\textbf{消元}的目的.
    \item   解所得的一元方程.
    \item   把得到的一个未知数的值,代入原方程
    组中的任一个方程,求出另一个未知数相应的
    值.    
\end{enumerate}
\end{blk}

这种求解的方法,叫做\textbf{加减消元法},简称\textbf{加减
法}.

\begin{ex}
    用加减消元法解下列方程组:
\begin{multicols}{2}
\begin{enumerate}
    \item $\begin{cases}
        x  +  y  = 8  \\ x  -  y  = 4  \\
    \end{cases}$
    \item $\begin{cases}
       3 x  +  2y  =21   \\ 2x  + 2 y  = 16  \\
    \end{cases}$
    \item $\begin{cases}
       2 x  +  \frac{y}{2}  = 2  \\ \frac{y}{2} =x +\frac{1}{2}       \\
    \end{cases}$
    \item $\begin{cases}
      7- 3 x =2  y    \\ -9+ 7 y  =5x   \\
    \end{cases}$
    \item $\begin{cases}
       0.1=\frac{x}{4} -\frac{y}{2}   \\ \frac{3x}{4}=\frac{1}{10}+\frac{5y}{2}   \\
    \end{cases}$
    \item $\begin{cases}
      5m-4n  = 33  \\3m+2n= 33  \\
    \end{cases}$
\end{enumerate}
\end{multicols}
\end{ex}

归根结底,解二元一次方程组的关键就是消元,
无论那种方法,在运用时都要注意:先由运算通性及
等式性质把方程组的每个方程进行变形,整理成为一
般形式:
\[\begin{cases}
    a_1x+b_1y=c_1\\
    a_2x+b_2y=c_2
\end{cases} \text{ ($a_1,b_1,c_1,a_2,b_2,c_2$都是已知数)}\]

    然后再根据特点,具体分析,灵活选用代入法或
加减法进行\textbf{消元}.方法力求简捷、方便.

\begin{example}
解方程组:
\begin{multicols}{2}
    \begin{enumerate}
    \item $\begin{cases}
4x+3y=7        \\
12x+9y=21   
    \end{cases}$
    \item $\begin{cases}
        4x+3y=7\\
        8x+6y=11
    \end{cases}$
\end{enumerate}
\end{multicols}
\end{example}

\begin{solution}
\begin{enumerate}
    \item 
    \begin{numcases}{}
        4x+3y=7        \\
        12x+9y=21  
    \end{numcases}
如果把(2.21)式乘以3,得
\begin{equation}
    12x+9y=21
\end{equation}
(2.23)$-$(2.22)可得:$0=0$.

这个结果并不奇怪,仔细观察原方程组,就会发
现,(2.22)式正是(2.21) $\x 3$的结果.这说明所给方程
组,实质上是一个二元一次方程,因此,可以断定:
\textbf{原方程组有无数多个解}.

\item \begin{numcases}{}
    4x+3y=7\\
    8x+6y=11
\end{numcases}
(2.24)$\x 2$可得:
\begin{equation}
    8x+6y=14
\end{equation}
(2.25)$-$(2.26)可得:$0=-3$,这是不可能的.

因此,原方程组就是\textbf{矛盾方程组,无解}.
\end{enumerate}
\end{solution}

通过这一例题,你能不能发现:方程组如果有无
数多解或无解,它们两个方程的各项系数之间应该具
有什么规律和特点?

\begin{ex}
解下列方程组:
\begin{multicols}{2}
\begin{enumerate}
    \item $\begin{cases}
        \frac{x}{3}-\frac{y}{7}=1\\
        1\frac{2}{3}x-\frac{5}{7}y=5
    \end{cases}$
    \item $\begin{cases}
        x=9-7y\\
        3x+y=19-20y
    \end{cases}$
\end{enumerate}
\end{multicols}
\end{ex}

学习了二元一次方程组的解法以后,在解决实际
问题时,就可以引入两个未知数,根据题目中的条
件,列出两个方程式,组成方程组.进而解出方程组,
使问题得到解决.

\begin{example}
    兄弟二人的语文考试成绩共190分,如果
知道兄是弟的成绩的90,那么,兄、弟各得多少
分?
\end{example}

\begin{solution}
    设弟弟得$x$分,兄得$y$分,则由题目所给条件可以得出:
\begin{numcases}{}
    x+y=190\\
    y=\frac{90}{100}x
\end{numcases}
将(2.28)代入(2.27):$x+\frac{90}{100}x=190$
解得 \qquad $x=100$

把$x=100$代入(2.28):$y=90$

答:兄得90分,弟得100分.
\end{solution}

\begin{ex}
    引入两个未知数,列方程组解:
\begin{enumerate}
    \item 某两数之和为100,之差为50.求两数.
    \item 班上男、女同学共52人,而女生人数的一半比男生总数少
      4人.求男、女生各几人?
\end{enumerate}
\end{ex}

\subsection{三元一次方程组及其解法}
\textbf{含有三个未知数的一次方程组},叫做\textbf{三元一次方
程组}.例如:

    方程组
\[\begin{cases}
    2x+3y+z=38\\3x+4y+2z=56\\4x+5y+z=66
\end{cases}\quad \begin{cases}
    x+y+z=2\\x-y+z=0\\x\qquad -z=4
\end{cases}\quad  \begin{cases}
    2m-R+2n=8\\ \qquad\;\; R+2n=-2\\ 3m+R-4n=1
\end{cases}   \]
等等,都是三元一次方程组.

    解三元一次方程组的关键,仍然是消元,其具体
方法仍是代入法和加减法.通过逐步消元,将三元转
化为二元,再转化为一元.每一次转化,都是运用
“通性”和等式性质.以保证解出一元方程的解,再
逐步求得三元一次方程组的解.

\begin{example}
    解方程组
\begin{numcases}{}
    2x+3y+z=38\\
    3x+4y+2z=56\\
    4x+5y+z=66    
\end{numcases}
\end{example}

\begin{analyze}
方程组中,未知数$z$的各项系数较简单,
    所以,可考虑先消去$z$,转化为二元一次方程组求解.
\end{analyze}

\begin{solution}
(2.31)$-$(2.29):$2x+2y=28$,即:
\begin{equation}
    x+y=14
\end{equation}
(2.29)$\x 2-$(2.30):    
\begin{equation}
    x+2y=20
\end{equation}
将(2.32)、(2.33)联立,由(2.33)$-$(2.32):$y=6$

将$y=6$代入(2.32):解出$x=8$.

把$x=8,\quad y=6$代入(2.29):解出$z=4$.

所以,原方程的解是:$(x,y,z)=(8,6,4)$
\end{solution}

\begin{rmk}
三元一次方程组的一个解,也是一个
\textbf{数值组},它由三个有顺序的数组成,不能忽视$(x, y, z)
=( 8,  6,  4 )$的写法.如果写成解集的形式就是:
\[\{(x,y,z)\}=\{(8,6,4)\}  \]
\end{rmk}

\begin{example}
解方程组:
\begin{numcases}{}
    x+y+z=2\\
    x-y+z=0\\
    x\qquad -z=4
\end{numcases}
\end{example}

\begin{analyze}
这个方程组的特点是:各个未知数的
系数绝对值在每个方程中都是1.因而,直接运用加
减法消元比较方便.
\end{analyze}


\begin{solution}
    (2.34)$-$(2.35):$2y=2$,$\therefore\quad y=1$

    (2.34)$+$(2.36):
    \begin{equation}
        2x+y=6
    \end{equation}

    把$y=1$代入(2.37):$2x+1=6$,$\therefore\quad x=\frac{5}{2}$

 把$x=\frac{5}{2}$代入(2.36):$\frac{5}{2}-z=4$,$\therefore\quad z=-\frac{3}{2}$

 所以,原方程组的解为:$(x,y,z)=\left(\frac{5}{2},1,-\frac{3}{2}\right)$,即解集为:
$$\{(x,y,z)\}=\left\{\left(\frac{5}{2},1,-\frac{3}{2}\right)\right\}$$
\end{solution}

\begin{example}
解方程组:
\begin{numcases}{}
    x+y\qquad =3\\
    \qquad y+z=5\\
x\qquad +z=4
\end{numcases}    
\end{example}


\begin{analyze}
这个方程组的特点是:每一个方程中
    都缺少一个未知数,且系数都是1,分布很均匀.用
    消元法可以灵活处理.
\end{analyze}

\begin{solution}
(2.38)$+$(2.39)$+$(2.40):$2x+2y+2z=12$,即:
\begin{equation}
    x+y+z=6
\end{equation}
应用(2.41)$-$(2.39),(2.41)$-$(2.40),(2.41)$-$(2.38)就可立即得出:
\[x=1,\quad y=2,\quad z=3 \]

所以,原方程组的解为:$(x,y,z)=(1,2,3)$,即解集为:
\[\{(x,y,z)\}=\{(1,2,3)\}  \]    
\end{solution}

引入三个未知数,列出三个一次方程,也可以解
一些实际问题.
\begin{example}
   某中学应届毕业生总共有$a$人,其中升入
高校和升入中专的占总人数的60\%;参加工作的正好
与升入高校的人数一样多.试问:升入高校、升入中
专和参加工作的各占多少人? 
\end{example}

\begin{solution}
    设升入高校$x$人,升入中专$y$人,参加工作
的$z$人,则由题目所给条件,可以列出:  
\begin{numcases}{}
    x+y+z=a\\
    x+y=\frac{60}{100}a\\
    x=z
\end{numcases}

解这个方程组,(2.42)$-$(2.43):$z=a-\frac{60}{100}a$,$\therefore\quad z=\frac{2}{5}a$

把$z=\frac{2}{5}a$代入(2.44):$\therefore\quad x=\frac{2}{5}a$

把$x=z=\frac{2}{5}a$代入(2.42):$y=a-\frac{2}{5}a-\frac{2}{5}a$,$\therefore\quad y=\frac{1}{5}a$.

因此:这个班有$\frac{2}{5}a$人升入高校,有$\frac{1}{5}a$人升入中专,有$\frac{2}{5}a$人参加了工作.
\end{solution}

\begin{ex}
    解下列三元一次方程组
\begin{multicols}{2}
\begin{enumerate}
    \item $\begin{cases}
        x+y-z=3\\-x+y+z=6\\x-y+z=4
    \end{cases}$
    \item $\begin{cases}
        x+2y+3z=6\\ 2x+3y+z=6\\3x+y+2z=6
    \end{cases}$
    \item $\begin{cases}
        2x+y=2\\2y+z=7\\x+2z=3
    \end{cases}$
    \item $\begin{cases}
        x+y+z=3\\5x-y+z=9\\x-2y+7z=24
    \end{cases}$
\end{enumerate}
\end{multicols}
\end{ex}

\section*{习题2.3}
\addcontentsline{toc}{subsection}{习题2.3}
\begin{enumerate}
 \item 已知方程$3x-y=1$,试先用$x$的算式表示出$y$来,再按下
表给出的$x$值求出$y$的相应的每一个值:
\begin{center}
    \begin{tabular}{c|ccccccccc}
   $x$ & $\cdots$ &$-3$&$-1$&0&$\frac{1}{2}$&$\frac{1}{3}$&1&2& $\cdots$\\
   \hline
   $y$ & $\cdots$ &&&&&&&&    $\cdots$\\
    \end{tabular}
\end{center}
    \item 试求不定方程$3x+y=10$的正整数解.
    \item 如果有10吨货物,要用马车和汽车一次运走,马车三辆
可装一吨,汽车一辆能装3吨.试问:如何派车?有几种方
案?
\item 判断下边方括号中给出的数值组$(x, y)$,是不是以下方程
组的解?
 \[\begin{cases}
    2x-y=7\\x+2y=-4
 \end{cases}\]   
\[[(-1,9),\quad  (1,-5),\quad  (0,-2),\quad  (2,-3),\quad  (3,-1),\quad  (-2,-1)] \]

\item 已知不定方程$my-2x=7$的一个解是:$(x,y)=(1,3)$,
试求$m$的值,并求出有一个未知数取0时,这个不定方程的
两个解.
\item 用代入法解下列二元一次方程组:
\begin{multicols}{2}
\begin{enumerate}
    \item $\begin{cases}
        7x+5y=6\\ y=x+3
    \end{cases}$
    \item $\begin{cases}
        a+2b=0\\ 3a+4b=6
    \end{cases}$
    \item $\begin{cases}
        7y-3=-11x\\ y=2x
    \end{cases}$
    \item $\begin{cases}
x=\frac{5}{2}y\\ 3x-4y=-14        
    \end{cases}$
    \item $\begin{cases}
        x+2y-1=0\\ x-4y+22=0
    \end{cases}$
    \item $\begin{cases}
        3x=\frac{y+1}{2}\\ 3x=\frac{y}{4}
    \end{cases}$
    \item $\begin{cases}
        \frac{1}{2}y-\frac{x}{4}=-\frac{5}{4}\\y=3(x-2)
    \end{cases}$
    \item $\begin{cases}
        3x-2y=5\\ x+4y=18
    \end{cases}$
    \item $\begin{cases}
        m+2n=4\\5m-1=4
    \end{cases}$
    \item $2(a+b)=3b+\frac{1}{2}=3a+b$
\end{enumerate}
\end{multicols}

\item 用加减法解下列方程组:
\begin{multicols}{2}
    \begin{enumerate}
        \item $\begin{cases}
            3x-2y=5\\ x+2y=11
        \end{cases}$
        \item $\begin{cases}
           2x+3y=11\\ y-2x=1 
        \end{cases}$
        \item $\begin{cases}
            5x+3y=7\\-5x+7y=3
        \end{cases}$
        \item $\begin{cases}
            2x+5y=25\\4x+3y=15
        \end{cases}$
        \item $\begin{cases}
            25x+17y=40\\17x+25y=40
        \end{cases}$
        \item $\begin{cases}
            5x=1+3y\\3x+7y=27
        \end{cases}$
        \item $\begin{cases}
            3x+5y=19\\ 8x-3y=67
        \end{cases}$
        \item $\begin{cases}
            7x-3y=10\\  3\frac{1}{2}x-50=6y
        \end{cases}$
        \item $\begin{cases}
            6.2v-4s=0.2\\ 8v-5s=1
        \end{cases}$
        \item $\begin{cases}
            3(x-1)=4(y-4)\\5(y-1)=3(x+5)
        \end{cases}$
        \item $\begin{cases}
            2(x+1)-3(y-1)=10\\ 2(x+1)+7(y-1)=20
        \end{cases}$
        \item $\begin{cases}
            \frac{y}{2}+\frac{z}{3}=13\\ \frac{y}{3}-\frac{z}{4}=3
        \end{cases}$
        \item $\begin{cases}
            \frac{m+n}{3}-\frac{m-n}{4}=+0.35\\ \frac{m+n}{3}+\frac{m-n}{2}=0.5
        \end{cases}$
        \item $0.1x-2=y+7=0.7x+y$
        \item $2(t-1)-s=t-1=5s$
    \end{enumerate}
\end{multicols}

\item 解下列方程组:(任选方法)
\begin{multicols}{2}
    \begin{enumerate}
\item $\begin{cases}
x=16-4y\\ y=34-4x
\end{cases}$
\item $\begin{cases}
x+y=7\\x:5=y:2
\end{cases}$
\item $\begin{cases}
\frac{2(x-0.5)}{5}-\frac{3(y+2)}{4}=1\\
3(2x-1)+15(y+2)=5
\end{cases}$
\item $\begin{cases}
\frac{v}{3}-\frac{t}{5}=\frac{2}{15}\\
\frac{2v}{3}+\frac{t}{5}=\frac{13}{15}
\end{cases}$
\item $\begin{cases}
3x-2y=7\\ 6x-4y=14
\end{cases}$
\item $\begin{cases}
x+2y=-1\\ x+y=\frac{1}{2}x+3
\end{cases}$
    \end{enumerate}
\end{multicols}

\item 如果已知方程组
\[\begin{cases}
  (m+1)x-(n-3)y  =-11\\
  mx+(n+2)y=7 
    \end{cases}\]
的一个解是$(x,y)=(1,-2)$,试求$m,n$的值.

\item 如果遇到三个方程式组成的二元一次方程组时,可以
先就其中任两个方程求出解答,再将解答代入第三个方程中
去检验.假如所求解答也满足第三个方程,那么,这个解就
是原方程组的解;假如不满足,那么,原方程组就无解.试
用这种方法解下列方程组:
\begin{multicols}{2}
    \begin{enumerate}
\item $\begin{cases}
4x-3y=1\\4x+y=5\\x-y=0
\end{cases}$
\item $\begin{cases}
4(x+y)+3=y+2\\ 7x-10=1-y\\ x+y=-1
\end{cases}$
\item $\begin{cases}
    3(2x+y)-1=2\\ x+1=2\\ y-3=2
    \end{cases}$
    \end{enumerate}
\end{multicols}

\item 解下列三元一次方程组:
    \begin{enumerate}
\item $\begin{cases}
    x+y+2z=2\\ 3x-y-4z=5\\ 2x+3y-2z=0
    \end{cases}$
\item $\begin{cases}
    x-y=\frac{3}{2}\\x+z=\frac{7}{10}\\y-z=-\frac{6}{5}
    \end{cases}$
    \item $\begin{cases}
    x+y=x+2y-3=z\\ 2x-3y+2z=5
    \end{cases}$
    \item $
    x+y+z=y-8x+107=x-2(x+y)+84=108$
    \end{enumerate}

\item 把下列方程组中的字母$t$当作已知数,试解出下列方
程组:
\begin{multicols}{2}
    \begin{enumerate}
\item $\begin{cases}
    x+y+t=3\\ x-y+3t=-1
\end{cases}$
\item $\begin{cases}
    2x+3y=1+t\\ y=x-t
\end{cases}$
\end{enumerate}
\end{multicols}

\item 如果$a,b$都是已知数,并且$|a|\ne |b|$,试解下面的字
母系数方程组:
\[\begin{cases}
    ax+by=b\\ bx+ay=a
\end{cases} \]
\end{enumerate}


\section{解应用问题}
    前面已经知道,用代数方法解决实际应用问题的
时候,先要做好充分准备,打好基础,就是先要引入
适当的未知数,根据题目中的数量关系直截了当地列
出方程或方程组;然后再去运用由数系运算通性及等
式性质而得出来的系统解法,求出符合题意的解答.

    本节将集中讨论如何灵活地解决各种一次方程
  (或方程组)的应用问题.

    
\begin{example}
    小明买练习本、生字本共15本,总共用去
一元四角钱.如果练习本的单价是一角六分,生字本
的单价是六分.试问:小明买练习本、生字本各多少
本?
\end{example}

\begin{analyze}
 方程就是一种等式.要列出符合题意
    的方程,就要先明确问题中的\textbf{等量关系}.这个问题
    中,有以下两个基本的等量关系:
\begin{enumerate}
    \item 总共15本,即 \[(\text{练习本数})+(\text{生字本数})=15 \]
    \item 总共用钱1.40元,即
    \[(\text{练习本数})\x(\text{单价})+(\text{生字本数})\x(\text{单价})=1.40\]
\end{enumerate}
因此,可以有两种考虑方法:
   
\textbf{解法1:} 引入一个未知数

设小明买练习本$x$本,则利用关系1可知,买生字本$(15-x)$本,
再利用关系2就可列出方程:
   \[x\cdot (0.16)+(15-x)\cdot(0.06)=1.4\] 
即:
\[\begin{split}
    0.16x+0.9-0.06x &= 1.4\\
             0.1x&=0.5\\          
    x&=5\text{(本)}
\end{split}\]
\[15-x=15-5=10\text{(本)}\]

答: 小明买了练习本5本,生字本10本.


\textbf{解法2:} 引入两个未知数

设小明买练习本$x$本,生字本$y$本.
则由关系式1与2,可以分别得出:
\begin{numcases}{}
    x+y=15\\
    x\cdot 0.16+y\cdot 0.06=1.4
\end{numcases}
解这个二元一次方程组,就可得出:
$(x,y)=(5,10)$

答: 小明买了练习本5本,生字本10本.
\end{analyze}

由此可见,同一个问题,有时可以引入一个未知
数,列出方程求解;也可以引入两个未知数,列出方
程组求解.两种解法各有利弊:一元方程求解简便,
但列方程较难;二元方程组列方程较易,但求解稍
繁.因而,在应用中可以灵活选择,不必要求一律.
一般来说,要选择“便于求解、引入未知数较少”的
方法.

\begin{ex}
    引入恰当的未知数,解下列各题:
    \begin{enumerate}
        \item 鸡兔同笼,共有头12个,有脚36只.问:笼中有鸡、兔各
  几只?
\item 若干学生平分若干支铅笔:如果每人5支,最后还多余3
支,如果每人7支,最后又缺5支.试问:有多少学生?
有多少支铅笔?
    \end{enumerate}
\end{ex}

\begin{example}
兄弟二人,从他们的家出发走同一条路线,
前往天安门广场.哥哥平均每小时走5公里,弟弟平
均每小时走3公里,假如哥哥比弟弟晚出发一小时,
却早到12分钟.试问:他们家到天安门广场有多远?
\end{example}

\begin{analyze}
    这是一个“行程问题”,涉及到的等
量关系是:
\[\text{路程}=\text{速度}\x\text{时间}\]
也可以写成:另外两种形式的关系:
\[\begin{split}
    \text{路程}\div \text{时间}&=\text{速度}\\
    \text{路程}\div \text{速度}&=\text{时间}
\end{split}\]
题目中,不仅已知兄弟二人的速度,还已知兄弟二人走
完这段路程的时间差.因此,只要选择适当的未知数,
不难由基本等量关系列出方程式来,把问题解决的.
\end{analyze} 

\textbf{解法1:} 引入直接未知数

设他们的家到天安门广场的路程为$x$公里,由
关系式$\text{时间}=\text{路程}\div \text{速度}$,就可以得出:
哥哥走到天安门广场所用时间为$\frac{x}{5}$小时;
弟弟走到天安门广场所用时间为$\frac{x}{3}$小时.

又由题意可知:哥比弟晚出发12分钟,这就是说:哥比弟少用$1\frac{12}{60}$小时,因此应有等量关系:
\[\text{哥所用时间}+1\frac{12}{60}=\text{弟所用时间}\]
即:
\[\begin{split}
    \frac{x}{5}+1\frac{12}{60}&=\frac{x}{3}\\
    \frac{x}{5}+\frac{6}{5}&=\frac{x}{3}\\
    2x&=18\\
x&=9\text{公里}
\end{split}\]
答:他们家到天安门广场的距离为9公里.

\textbf{解法2:} 引入间接未知数

设弟弟到天安门广场总共用了$t$小时,则由题意可知:
哥哥用了$\left(t-1\frac{1}{5}\right)$小时.因此,由基本关系式
$\text{路程}=\text{速度}\x\text{时间}$ 可以知道:     
\[\begin{split}
 \text{哥哥所走路程}  &=5\x \left(t-1\frac{1}{5}\right) \text{(公里)}\\
 \text{弟弟所走路程}&=3\x t \text{(公里)}
\end{split}\]

又知,哥、弟二人走的是同一路线,所以:
\[\begin{split}
    5\left(t-1\frac{1}{5}\right)&=3t\\
    2t&=6\\
    t&=3\text{(小时)}
\end{split}\]

这就是说:弟弟用了三小时走到天安门广场.

由此,再利用基本关系式可以得到:他们家到天
安门广场的路程为:$3\x3= 9$(公里)
    
答: 所求路程是9公里.

从这里可以看出,在解决应用问题时,对于同一
个间题,有时可以从不同的角度引进未知数(可以
直接设所求的量为未知数,也可以间接设一个与所求
量有关的未知数).由于引用未知数的不同,同一个
问题完全可以列出不同形式的方程.但最终所求的量
的答数是一样的.这真是“异途同归”了.


\begin{ex}
    解下列应用题:
\begin{enumerate}
    \item 甲、乙两地相距165里;小张以每小时30里的速度骑自行
    车由甲地去乙地,小李以每小时25里的速度骑自行车由乙
    地去甲地.
    
    问:他们两人如果同时出发,几小时后相遇?
    在距甲地多少里的地方相遇?
    \item 甲、乙两个车站相距284公里,甲站有一列快车以每小时70
    公里的速度要开往乙站;而乙站有一列慢车以每小时48公
    里的速度要开往甲站.如果慢车先开车一小时.
    
    试问:快车开
    几小时后两车才能相遇?在距离甲站多少公里的地方相遇?
\end{enumerate}
\end{ex}    
    
通过以上例题,可以归纳出解应用问题的一般步
骤是:

\begin{blk}{}
\begin{enumerate}[I. ]
    \item 审清题意,明确数量关系.
    \item 引入适当的未知数$x$.并注意写明未知
    数所代表的量的单位.如:$x$公里,$x$小时等.
    \item 用含有未知数的算式,表示其它有关的
    量数.
    \item 由基本关系式,列出方程或方程组(一
    般说:列出的方程个数与所设未知数的个数是相
    等的).
    \item 解方程(或方程组).
    \item 检验所求解是否符合题意后,写出答
    案.
\end{enumerate}
\end{blk}

必须指出:应用题的检验是必要的,一方面可以
检查计算是否正确,另一方面还可以检验所求方程的
解是否符合题意,不符合题意的要舍去.

\begin{example}
    两地相距28公里,小明以每小时15公里的
速度,小亮以每小时30公里的速度,分别骑自行车和
开汽车从同一地前往另一地.如果小明先出发一小
时,试问:小亮几小时以后,才能乘汽车追上小明?
\end{example}

\begin{solution}    
设小亮开车$x$小时能追上小明,
        则小亮所行路程是:$30x$(公里).

    这时,先出发一小时的小明,已经走过的路程
为:$15(x+1)=15x + 15$(公里)

    小亮要追上小明,必须有关系式:
            $  30x=15x+15$

$\therefore\quad x= 1$(小时)

    这就是说:小亮开车一小时,就能追上小明.

\textbf{检验:} $x=1$,虽然能使方程$30x =15x + 15$
成立,但这时,小亮开车走出30公里的路程,而两地
实有距离只有28公里,由此可见,在题目所给出的两
地之间,小亮是没有追上小明的.$x=1$不符合题意,
应舍去.

    答:在此两地之间,小亮追不上小明.
\end{solution}


\begin{ex}
    解下列“追及问题”
\begin{enumerate}
    \item 甲、乙两人从同地出发练习短距离赛跑,甲每秒跑7米,
    乙每秒跑6.5米.如果甲让乙先跑一秒钟,试问:甲经过
    几秒钟才能追上乙?
    \item    甲、乙两人同时从同地出发,绕400米一圈的跑道赛跑,
    如果甲跑一圈要一分钟,乙跑一圈要一分零20秒,试问:
    甲、乙出发几分钟以后,才能再相遇?
\end{enumerate}  
\end{ex}



\begin{example}
一只船在两个码头之间航行.顺水时需要
4.5小时,逆水返回需要$5\frac{1}{3}$小时.如果水流速度是每
小时1公里.试问:这两个码头相距多少公里?
\end{example}

\begin{analyze}
    这种应用问题,实际上仍是“行程问
题”.如果能求出这只船在顺水时的速度或在逆水中
的速度,那么,两码头的距离就立即可以求出来,题
目中只有水流速度.但只要注意:船在静水中的速度
与水流速度、逆水速度、顺水速度有以下基本关系,
问题就容易解决了.
\[\begin{split}
    \text{顺水速度}&=\text{船在静水中的速度}+\text{水速}\\
    \text{逆水速度}&=\text{船在静水中的速度}-\text{水速}\\
\end{split}\]    
\end{analyze}

\begin{solution}
设船在静水中的速度是$x$公里/小时.
则顺水速度为$(x+1)$公里/小时,

因此,船在顺水中,行驶的距离为:
      $(x+1) \x 4.5$ (公里).

又逆水速度为$(x-1)$公里/小时,因此船在逆水返回时,所行驶的路程为:
$(x-1)\x5\frac{1}{3} $(公里).
    
由于顺水、逆水都是航行于两码头之间,因此:
\[(x+1)\x 4.5=(x-1)\x 5\frac{1}{3}\]
解这个方程:    
\[\begin{split}
    27 (x+1)&=32 (x-1)\\
(32-27)x&=27+32\\
5x&=59\\
x&=11\frac{4}{5}\text{(公里/小时)}
\end{split}\]
$\therefore\quad 4.5 (x+1) =4.5(11.8+1)=57.6$(公里)

答:两码头之间相距57.6公里.
\end{solution}

\begin{ex}
\begin{enumerate}
    \item 汽船从甲地顺流开往乙地,所用的时间比逆流由乙地开回
    甲地所用时间少1.5小时,扣果这只汽船在静水中速度是
    20公里/小时,水流速度是3公里/小时.试问:甲、乙两
    地相距多少?
    \item $A, B$两码头相距若千公里,某船由$A$码头顺水行至$B$码头
    要3小时,回来时多用半小时,如果知道船在静水中的速
    度是26公里/小时,试求水流的速度?
\end{enumerate} 
\end{ex}

\begin{example}
 有一水池,用两台水泵抽水.如果单开甲
    泵,5小时抽完这一池水;如果单开乙泵,2.5小时
    才能抽完.试问:
\begin{enumerate}
    \item 两泵同时开,几小时能把水抽完?
    \item 如果甲泵先抽2小时,剩下的再单用乙泵来
抽,还需要多少时间才能抽完?
\end{enumerate}
\end{example}

\begin{analyze}
这是应用问题中常见的“工程问题”.
常把整个“工程的量”看作“1”进行分析.

    在这个问题中,如果把“一池水”的总量看作整
体“1”,那么,由题意可知:甲泵一小时的抽水量
为$\frac{1}{5}$;乙泵每小时的抽水量是$\frac{1}{2.5}$;因此甲、乙两
泵一齐开,每小时的抽水量就是$\left(\frac{1}{5}+\frac{1}{2.5}\right)$.这样,
再利用基本关系:
\[\text{每小时的工作量} \x \text{工作时间}=\text{总工作量}1\]
就可以解决问题了.
    
\end{analyze}

\begin{solution}
\begin{enumerate}
    \item 设两泵同时开,$x$小时可以抽完这池水,因此就有:
\[\begin{split}
    \left(\frac{1}{5}+\frac{1}{2.5}\right)x&=1\\
    \frac{3}{5}x&=1\\
    x&=\frac{5}{3} \text{(小时)}
\end{split}\]
这就是说,两泵合开,$1\frac{2}{3}$时(1小时40分钟)
可以抽完这一池水.

    经检验,符合题意.

    答:两泵合并,1小时40分钟可以抽完.

\item  设乙泵再抽$x$小时将水抽完,
因此,依题意就有:
\[\begin{split}
    \frac{1}{5}\x 2+\frac{1}{2.5}\x x&=1\\
    \frac{1}{2.5}x&=\frac{3}{5}\\
    x&=1.5 \text{(小时)}
\end{split}\]

答:甲泵抽2小时后,乙泵再抽1.5小时,就可以抽完这一池水.
\end{enumerate}
\end{solution}




\begin{ex}
\begin{enumerate}
    \item 检修一台机器,甲班需要7.5小时完成,乙班需要5小时
    完成.试问:
    \begin{enumerate}
        \item 两班同时参加检修,要几小时才能完成?
        \item 如果甲班先做4小时,其余的工作由乙班完成,
    还要多少时间才能完成?
    \end{enumerate}
    
    \item 某工程,甲队需要20天完成,乙队需要30天完成,丙队需
    要40天完成;如果甲、乙、丙三队联合工作8天以后,其
    余工程由甲队单独完成.试问:甲队还需要几天完成?
\end{enumerate}
\end{ex}

\begin{example}
浓度为95\%的盐水有600克,问:
\begin{enumerate}
    \item 只要再加多少水,就能稀释成浓度为75\%的盐水?
    \item 只要再加多少盐,就能得到浓度是98\%的盐水?
\end{enumerate}
\end{example}


\begin{analyze}
这是应用题中的“浓度问题”(即百
分数问题).只要明确基本关系:
\begin{enumerate}
    \item $\text{浓度(百分比)}=\frac{\text{溶质}}{\text{溶质}+\text{溶剂}}$
    \item 增加溶剂,可使浓度减小,但其中的溶质总量不变.
    \item 增加溶质,可使浓度加大,但其中的溶剂总量不变.
\end{enumerate}
这一类应用问题是不难解决的.
\end{analyze}

\begin{solution}
\begin{enumerate}
    \item  设加水(溶剂)$x$克,
    则加水前,95\%的盐水600克中,含有纯盐(溶
质)的总量是: $600 \x 95\%$克

    加水后,75\%的盐水$(600+x)$克中,含有纯盐的
总量是: $(600+x) \x 75\%$克.

    因此,由分析中的关系2可知:
            \[  600 x 95\%=(600+x) \x 75\%\]
即:$8\x95=600+x$
$\therefore\quad x=160$(克)

答:需要加水160克,就得到浓度是75\%的盐水.
\item  设加盐(溶质)$y$克,
则加盐前,95\%的盐水600克中,含有水量:
        $600\x (1-95\%)$克

加盐后,98\%的盐水$(600+y)$克中,含有水量:
$(600+y) \x (1-98\%)$克.

因此由分析中的关系3可知:
        \[600 \x (1-95\%)=(600+y) \x (1-98\%)\]
    即:\[\begin{split}
        600\x5\%&=(600+y) \x 2\%\\
        300\x 5&=600+y\\
    y&=900 \text{(克)}
    \end{split}\]

    答:只要加盐900克,就可得到浓度为98\%的盐
水.
\end{enumerate}       
\end{solution}

\begin{example}
实验室里只有浓度为8\%的盐水和浓度为
5\%的盐水;现在需要把两种浓度的盐水混合起来,
制成浓度为6\%的盐水300克,试问:这两种浓度的
盐水各用多少克才能合适?    
\end{example}

\begin{analyze}
这同样是浓度问题.解决这类稍复杂
的混合溶液问题的关键是要抓住:\textbf{混合前、后的总重量不变;混合前、后纯溶质(盐等)的重量不变;混
合前、后纯溶剂(水等)的重量当然也不变}等关系,
引进未知数,列出方程式.

    由此,对本题可作如下列表分析:
\begin{center}
\begin{tabular}{|c|c|c|c|}
\hline
溶液  & \multicolumn{2}{c|}{混合前} &混合后\\
\hline
浓度 & $\quad8\%\quad$  & 5\% & 6\%\\
\hline
总重量& \multicolumn{2}{c|}{$x\text{克}\quad +\quad y\text{克}$} &300克\\
\hline
纯盐重  & \multicolumn{2}{c|}{$(x\cdot 8\%+y\cdot 5\%)\text{克}$} &$300\x 6\%$克\\
\hline
\end{tabular}
\end{center}
 \end{analyze}


\begin{solution}
设需用浓度8\%的盐水$x$克,需用浓度5\%的盐水$y$克.
    则由题意可得:
\[\begin{cases}
    x+y=300\\
    x\x 8\% +y\x 5\%=300\x 6\%
\end{cases}\]
    解这个二元一次方程组,可得:
           \[(x,y)=(100,200) \]

    答:用100克浓度为8\%的盐水和200克浓度为500
的盐水,混合后就可得到浓度为6\%的盐水300克.

\textbf{注:} 引入一个未知数也可以解这一题,试试
看.
    
\end{solution}

\begin{ex}
\begin{enumerate}
    \item 含糖15\%的糖水20斤.
    \begin{enumerate}
        \item 要想得到含糖10\%的糖水,需要加多少斤的水?
        \item 要想得到含糖24\%的糖水,需要蒸
发掉多少斤水?
    \end{enumerate} 
\item 两种铜块,分别含铜64\%和80\%.试问:这两种铜块各取
多少克熔化以后,才能得到含铜74\%的铜块500克?
\end{enumerate}
 \end{ex}



\begin{example}
(我国古代问题):上禾三束、中禾二束、
下禾一束,共有禾三十九斗;上禾二束、中禾三束、
下禾一束,共有禾三十四斗;上禾一束、中禾二束、
下禾三束,共有禾二十六斗.

问:上、中、下禾每束
各有禾几斗?
\end{example}

\begin{note}
    上、中、下禾,就是“上等谷物、中
    等谷物、下等谷物”;一束,就是“一捆”;斗,是
    古代一种衡器,可以用来量谷物的容积.
\end{note}

\begin{solution}
    设上禾每束有禾$x$斗,
    中禾每束有禾$y$斗,
    下禾每束有禾$z$斗.
    则依题意可得出下列各方程:
\[\begin{cases}
    3x+2y+z=39\\
    2x+3y+z=34\\
    x+2y+3z=26
\end{cases}\] 
解这个三元一次方程组,可以得出:
\[ (x,y,z)=\left(9\frac{1}{4},\; 4\frac{1}{4},\; 2\frac{3}{4}\right)\]

答:上禾每束有禾$9\frac{1}{4}$斗;
中禾每束有禾$4\frac{1}{4}$斗;
下禾每束有禾$2\frac{3}{4}$斗.
\end{solution}



\section*{习题2.4}
\addcontentsline{toc}{subsection}{习题2.4}

适当选择未知数,列方程解下列问题:
\begin{enumerate}
    \item 甲煤场有煤2000吨,每天运出15吨;乙煤场有煤800吨,
每天运进25吨.几天后两煤场的煤存量相同?
\item 有一个仓库存货200吨,每天运出25吨;第二仓库存货
80吨,每天运进5吨.几天后第一仓库存货是第二仓库存货
的三倍?
\item  $A, B$两油槽中,分别有油480升、180升,要使$A$中的油
是$B$中的二倍,须从$A$中抽出多少油给$B$才行?
\item 生产队现有水田108亩,旱地54亩.要使全队的旱地只
占水田的20\%,还应把多少亩旱地改为水田?
\item 一面靠墙,其余三面要用350米的篱笆围成一个长方形
养猪场,要使它的长是宽的2倍,应如何围法?
\item 长方形的长与宽之比为5:2,周长28尺,试求这个长方
形的面积.
\item 一桶煤油毛重8斤,从桶中取出煤油一半以后,毛重4.5
斤,问:这一桶煤油有多少?桶重多少?
\item 小华现已读数学书41页,英语书28页.她的计划是每天
读数学书5页、英语书4页.问:几天后,她读的数学书页
数是英语书的两倍?
\item 甲、乙两人共有图书78本.如果甲赠给乙5本以后,两
人的图书就一样多.问:甲、乙原来各有几本?
\item 一堆水果,如果用每筐能装45斤的小筐来装,可以装
40筐,如果改用大筐装,每筐可比小筐多装$\frac{1}{10}$.问:这堆水
果能装几大筐?
\item 一个班共有学生52人,其中共青团员、少先队员、同
学的人数比为2:10:1.问:这个班应有团、队员和同学各几
人?
\item 甲、乙两地相距162公里,甲地有一列慢车,每小时可
以开48公里,乙地有一列快车,每小时可以开60公里,试
问:
\begin{enumerate}
    \item  两火车同时相向而行,多少时间可以相遇?
    \item  两车同时相背而行,几小时以后,两车相距270公
里.
\item  如果两车相向而行,慢车先开出一小时,再用多少
时间两车才能相遇?
\item  两车同时同向而行(快车在后边).几小时后,快
车可以追上慢车?
\item  两车同时同向而行(慢车在后边).几小时后,两
车相距200公里?
\end{enumerate}
 
\item  卡车以每小时30公里的速度由甲地开往乙地,半小时
以后,小轿车以每小时40公里的速度也从甲地开往乙地,结
果两车同时到达.问:甲、乙两地相距多少公里?
\item  甲、乙两运动员在四百米环形跑道上竞走,甲每分钟
走150米,乙每分钟走100米.如果两人同时由同一起点出
发,问:几分钟以后,甲才能再一次与乙相遇?
\item  买进水果若干筐,每筐原价3元.如果按照每筐四元
的价钱卖出,那么,卖出全部水果的一半又10筐时,就已经
收回了全部水果的成本.问:一共买进水果多少筐?
\item  轮船在静水里的速度是16公里/小时,由甲地顺水航行
到乙地要7小时,逆水返回要9小时,试求甲、乙两地的距
离.
\item  如果风速是3公里/小时,汽车速度是20公里/小时.
要求这一汽车由$A$地顺风开往$B$地比逆风开回来少用1.5小
时,试求:$A,B$两地的距离.
\item  一件工作,甲单独做5小时完成,乙单独做8小时完
成.问:两人合做几小时完成?
\item  某项任务,师傅单独做4小时可以完成,徒弟单独做6
小时可以完成.如果先由徒弟单独做2小时以后,师、徒二
人再合做,试问:他们再做几小时,才能完成这一任务?
\item  某游泳池装有$A, B, C$三个进水管,若单独开放一管,
分别要45分钟、1小时、1.5小时才能注满.问:三管一齐
开,需要几小时才能注满?
\item  甲、乙、丙三个生产队合修一条水渠,计划用52人,
按照各队收益面积的比3:4:6摊派劳动力.问:各队应派几
人?
\item  100个工人,100一台机器,老工人一人看管3台,徒工
3个人看管一台.问:有多少老工人?有多少徒工?
\item  现有含氮16\%的氨水30斤.要配置成0.15\%的氨水对
菜地追肥.问:需要加多少水?
\item  现有含盐8\%和20\%的两种盐水,要配置成含盐12\%
的盐水100公斤,两种盐水应各取多少?

\item  在含酒精20\%的液体中,加入10公斤的水,就变成含
酒精16\%的液体,试求:原来的液体有多少公斤?
\item  两块合金,一块含铜90\%,一块含铜80\%.现在要把
两块熔合在一起,得到含铜82.5\%的合金240克,问:应各
取多少克?

\item 锡的比重是7.3${\rm g}/{\rm cm}^3$,铅的比重是11.3${\rm g}/{\rm cm}^3$,现在要想得到铅锡合金166g,它的比重是8.3${\rm g}/{\rm cm}^3$.试问:铅、锡应各取多少克?

\item 一个容器装49升水,另一个容器装56升水,如果把第
二个容器的水倒满第一个容器,那么第二个容器还剩下的水
相当于这个容器容量的一半,如果把第一个容器的水倒满第
二个容器,那么第一个容器还剩下的水相当于这个容器的容
量的三分之一.求这两个容器的容量各是多少?
\item $M, N$两地相距24公里,公共汽车和直达快车在8点
45分准时从$M$和$N$迎面开出.这两辆车平常都在8点52分相
遇.有一次,直达快车晚开8分钟,就在8点57分与公共汽
车相遇.试求这两种车的速度各是多少?

\end{enumerate}

将下列应用题,引入两个(或三个)未知数求解.

\begin{enumerate}
    \item  甲数的7\%与乙数的8\%一共是12;而甲数的8\%与乙
    数的7\%的差是4.5,求这两个数.
    \item 如果把长方形的长增加6米,宽减少3米,它的面积不
    变;如果长减少3米,宽增加2.4米,它的面积仍不变.试
    求长方形原来的长和宽各是多少?
    \item 甲、乙两班学生,开学初团员共有8人,各占全班人
    数的$\frac{1}{10}$、$\frac{1}{12}$,学期末团员共有18人,各占全班人数的$\frac{1}{4}$、$\frac{1}{6}$.
    问:两班各有学生多少人?
    \item 某人乘车以每小时40公里的速度可以按时由甲地到达
    乙地,但因出发时间晚了15分钟,汽车便以每小时50公里速
    度行驶,结果却比预定时间还提前3分钟到达乙地.试求:
    甲、乙两地之间的距离.
    \item 小王与小张从相距36公里的$A, B$两地相向而行.如果
    小王先动身2小时,则他们在小张动身两小时三十分钟后相
    遇;如果小张先动身2小时,则他们在小王动身3小时后相
    遇.试问:小张与小王每小时各走多少公里?
    \item 甲、乙二人同时绕400米的环形跑道行走.如果他们同
    时从同一起点背向而行,2,5分钟可以相遇;如果他们同时
    由同一起点同向而行,12.5分钟甲能追上乙,试求甲、乙两
    人每分钟各走几米?
    \item 一批零件共420个,如果甲先做2天,再与乙合做2天
    可以做完,如果乙先做2天,再与甲合做3天可以做完.试
    求:甲、乙两人每人每天各能做多少个零件?
    \item 有52人要担任运土、挖土两项任务,如果每人平均挖土
    6方或运土480.公斤(已知每方土重1.8吨),问:如何分
    配人力,才能使挖土、运土主作配套,不至于窝工?
    \item 一艘轮船载重量为520吨,容积为3000立方米.现有
    $A, B$两种货物,$A$货每吨有2立方米,$B$货每吨有8立方
    米.问:这艘船应装$A, B$两种货物各多少吨,才能充分利
    用它的船仓?
    \item 有三个数总和是15.第一个数与第二个数的差恰是第
    二个数与第三个数的差,而且第二、三两数的和比第一个数
    大2.求这三个数各是多少?
    \item 有人民币三种:二角、伍角、一元共37张,价值13元9
    角.如果已知二角票的张数是伍角票张数的二倍,问:三种
    票各有几张?
    \item (我国古代问题)今有上禾二束、中禾兰束、下禾四
    束都有谷不满一斗,若上禾取中禾、中禾取下禾,下禾取上
    禾各一束,则有谷都正好一斗,问:上、中、下禾一束各有
    谷几斗?
    
\end{enumerate}

\section*{本章内容要点}

一、这一章主要是应用数系运算通性及等式性
质,解一次方程和方程组,并且能解决相应的应用问
题.

\vskip 2ex 

    二、含有未知数的等式,叫做方程.能使方程式
两边相等的未知数的值,就是方程的解.

\vskip 2ex 

    三、含有一个未知数的方程,叫做一元方程,如
果只含有一个未知数,分母不含未知数,且最高次项
的指数是1的方程,叫做一元一次方程.解一元一次
方程的原理和方法是:由数的运算通性和等式性质,
归纳出去分母、去括号、移项变号、合并同类项、除
以未知数的系数等具体规则,应用这些具体规则,就
可以把方程的解求出来.

\vskip 2ex 

    四、含有两个未知数的一次方程,叫做二元一次
方程.二元一次方程的解是一个数值组,记作$(x, y)$;
任一个二元一次方程的解都有无限多个.

\vskip 2ex 

    五、两个以上的方程联合在一起,组成一个方程
组.因而,由几个含有两个未知数的一次方程所组成
的方程组二网做三元一次方程组.能够同时满足方程
组中所有各个方程的未知数所取的数值组,叫做这个
方程组的解.

\vskip 2ex 

    六、解多元一次方程组的关键是消元.具体方法
有:加、减消元法和代入消元法.

\vskip 2ex 

    这里应特别指出:加减法消元是较普遍且重要的
方法,其要点就是,应用等式的性质,将两个方程中
相同的某一个未知数的系数变形成为绝对值相等.然
后把方程的两边分别相加或相减,就可以消去这个未
知数.

    因此,可以说:解多元一次方程的过程,就是逐
步消元求解的过程.即
\begin{center}
\begin{tikzpicture}[>=latex, scale=1.3]
\node  (A) at (0,0) {\Large 多元} ;
\node  (B) at (2,0) {\Large ……} ;
\node  (C) at (4,0) {\Large 三元} ;
\node  (D) at (6,0) {\Large 二元} ;
\node  (E) at (8,0) {\Large 一元} ;

\draw[->] (A)--node[above]{消元}(B);
\draw[->] (B)--node[above]{消元}(C);
\draw[->] (C)--node[above]{消元}(D);
\draw[->] (D)--node[above]{消元}(E);

\end{tikzpicture}
\end{center}

\vskip 2ex 

七、解应用问题就是运用数学工具解决实际问
题,这就要:
\begin{enumerate}
    \item 审题:弄清题意,分析问题中涉及到的量与量
    之间的关系.
    \item 引入未知数,并用未知数表示出有关的量.
    \item 正确列出方程(或方程组).一般来说,引入
    的未知数个数与所列方程的个数是相等的.
    \item 准确地求出来知数的值,即所列方程(方程
    组)的解.
    \item 检验:所列方程(方程组)的解是否符合题
    意.将不合理的值舍去,从而写出正确答案.
\end{enumerate}

        其中,分析量之间的关系是列方程的关键;列方
    程是解决问题的基础;解方程又是解决问题的主要手
    段.这都是解应用问题时,必须注意的几个重要环节.

\section*{复习题二}
\addcontentsline{toc}{section}{复习题二}

\begin{enumerate}
    \item 如果已知汽车的速度是$a$公里/小时,摩托车的速度是$b$
    公里/小时 $(a>b)$.试求:
\begin{enumerate}
    \item 两车同时、同地、同向开出,5小时后,两车相距
    多少公里?
    \item 一个人坐汽车走了$m$小时,接着又坐摩托车走了,小
    时,一共走了多少公里?
    \item 两车都开出了10公里,汽车比摩托车少用几小时?
    \item 两车都开了$d$小时,摩托车比汽车少走多少公里?
\end{enumerate}

        \item 用图中已知的字母,表示出阴影部分的面积来:
\begin{center}
\begin{tikzpicture}[>=latex,scale=.7]
\begin{scope}
\draw[pattern=north east lines] (2,1.5) circle (2.5);
\draw[fill=white] (0,0) rectangle (4,3);
\draw[<->](0,0)--node[above]{$2R$}(4,3);
\node at (2,0)[above]{$a$};
\node at (0,1.5)[right]{$b$};

\end{scope}    
\begin{scope}[xshift=6cm]
    \draw[pattern=north east lines] (0,0) rectangle (5,3);
    \draw[fill=white]  (.5,0) arc (180:0:2);
    \draw [->] (2.5,0)--node[above]{$R$}+(40:2);
    \node at (2.5,3)[above]{$a$};
    \node at (5,1.5)[right]{$b$};
\end{scope} 
\begin{scope}[yshift=-6cm]
    \draw[pattern=north east lines]  (0,0) arc (180:0:2.6); 
    \draw[fill=white](2.6,1.3) circle(1.3);
    \draw (0,0)--(5.2,0);
    \draw[|<->|] (0,-.2)--node[below]{$D$}(5.2,-.2);
\end{scope} 
\begin{scope}[xshift=6cm, yshift=-6cm]

\fill  [pattern=north east lines] (0,0) rectangle (6,4);
\fill [white](0,0) rectangle (1,1);
\fill [white](0,3) rectangle (1,4);
\fill [white](5,0) rectangle (6,1);
\fill [white](5,3) rectangle (6,4);

\foreach \x in {0,4}
{
    \draw (1,\x)--(5,\x);
    \draw (\x*1.5, 1)--(\x*1.5, 3);
}

\draw (0,1)--(1,1)--node[left]{$x$}(1,0);
\draw (0,3)--(1,3)--node[left]{$x$}(1,4);
\draw (5,0)--(5,1)--(6,1);
\draw (5,4)--(5,3)--node[above]{$x$}(6,3);
\node at (3,4)[above]{$4x$};
\node at (6,2)[right]{$2x$};

\end{scope}     
\end{tikzpicture}
\end{center}    

\item 解下列方程式:
\begin{enumerate}
    \item $\frac{x}{2}-7-\frac{x}{5}=\frac{1}{5}-\frac{x}{4}+\frac{x}{20}  $
    \item $\frac{1}{3}\left(3x+\frac{10-7x}{2}\right)-\frac{1}{6}\left(2x+\frac{2x+2}{3}\right)=\frac{x}{2}-1$
    \item $3\left\{x-\frac{3x+1}{4}-\left[1-2\left(x-\frac{3-x}{5}\right)\right]\right\}=8x-6$
    \item $(x-4)(x+3)=(x+1)(x+2)$
    \item $(2x+3)(2x-3)=(2x-3)^2$
\end{enumerate}
提示:最后两题先用分配律展开.

\item \begin{enumerate}
    \item 在公式$V=\frac{\pi\cdot N\cdot D}{1000}$中,已知$V=120$,$D=80$,$\pi=3.14$,求$N$(保留两位小数).
    \item 如果球的体积公式是$V=\frac{4}{3}\pi R^3$,已知球的半径$R=20{\rm cm}$,体积$V=3.2\x 10^4{\rm cm}^3$.试计算$\pi$的值应取多少?
\end{enumerate}

\item 解下列字母系数的方程式:
\begin{enumerate}
    \item $mx+n=m(2x+1)\qquad (m\ne 0)$
    \item $mx+\frac{x}{m}=1\qquad (m\ne 0)$
    \item $\frac{x}{a}+\frac{1}{b}=2\qquad (a\ne 0,\quad b\ne 0)$
\end{enumerate}

\item 试求下列各题中的$k$值.
\begin{enumerate}
    \item $kx^2-12x-3=0$ 有一个根为1
    \item $x^2+kx+15=0$ 有一个根为$-\frac{1}{2}$
    \item $x^2+kx-2k=0$ 有一个根为$0$
    \item $x^2-2ax+k=0$ 有一个根为$a$
\end{enumerate}

\item 解下列方程:
\[|x-2|=1,\qquad |2x-4|+|x-2|=0\]

\item 求不定方程$3x+5y=15$的非负整数解.

\item 已知方程组$\begin{cases}
    mx+2y=n\\ 4x-ny=2m-1
\end{cases}$的解是$(x,y)=(1,-1)$,试求$m,n$的值.

\item 解下列方程组:
\begin{enumerate}
    \begin{multicols}{2}
        \item $\begin{cases}
        \frac{1}{2}x-y+1=0\\ 2\frac{1}{2}x+6y=28
    \end{cases}$
    \item $\begin{cases}
        \frac{x}{2}=\frac{y}{5}\\ 75\%x+40\%y=1.4
    \end{cases}$
    \end{multicols}
    
    \item $\begin{cases}
        \frac{2x-3y+1}{2}+\frac{3x-2y-3}{3}=1\\
        \frac{x+2y+6}{4}-\frac{4x+2y-2}{5}=0
    \end{cases}$
    \item $\frac{2x+y+6}{4}=\frac{4x-3y-7}{8}=\frac{-6x-7y+10}{8}$
    \item $2x-y+11=-7y+3z=7x+z+7=7$
    \begin{multicols}{2}
    \item $\begin{cases}
        \frac{y}{2}+\frac{z}{3}=x+2\\x-y=\frac{1}{2}-\frac{z}{4}\\\frac{x}{4}+\frac{y}{2}=\frac{z}{2}-\frac{7}{4}
    \end{cases}$
    \item $\begin{cases}
        x+y=3a\\y+z=4a\\z+x=5a
    \end{cases}$($a$是已知数)
    \item $\begin{cases}
        x+2y=2\\y+3z=1\\z-t=4\\x+y+z+t=-1
    \end{cases}$
    \item $\begin{cases}
        x+y+z+u=a\\ x+y-z-u=b\\ x-y+z-u=c\\x-y-z-u=d
    \end{cases}$
\end{multicols}
\end{enumerate}

\item 如果已知$3x-z=x+y+z=4x+2y-z$,试求:$x:y:z$
\begin{multicols}{2}
\begin{enumerate}
    \item $\begin{cases}
        x=t\\ y=2t^2
    \end{cases}$
    \item $\begin{cases}
        x=3-t\\ y-5=t+1
    \end{cases}$
\end{enumerate}
\end{multicols}

\end{enumerate}

\textbf{适当引入未知数,列出方程或方程组解下列各应用题:
}

\begin{enumerate}
    \item 如果有三个紧相连的整数之和为126,试求这三个整
    数.
    \item 如果三个相邻的偶数之和为114,求出这三个偶数来.
    \item 一个分数的分子是分母的$\frac{5}{8}$,且分子、分母之和为
    143,试求这个分数.
    \item 一个二位的自然数恰好等于它的两个数字之和的五倍,
    试求这个二位数是多少?
    \item 一个四位数$7abc$,如果把7调到最后一位上,就变成
    另一个四位数$abc7$,那么,这个四位数就比原来减少了864,
    试求这个四位数.
    \item 一个六位数$1abcde$的三倍,正好是六位数$abcde1$.试
    求这个六位数.
        
    \textbf{提示:} 这类数字问题,均可设未知部分为$x$,再由已知数字所在的位数,把这个数表示成含二的算式.如:

    第5题中,设$abc=x$,则$7abc=7000+abc = 7000+x$

    第6题中,设$abcde = x$,则$1abcde =100000 + x$等

 \item  一个长方形的长是宽的三倍,如果把宽增加3米,这个
 长方形就变成一个正方形了.试求这个长方形的面积是多少?
    
\item   一个正方形环(如图)面积
为56${\rm cm}^2$,这个环形各边的
宽都是2cm,试求这个环形的
外周长、里周长各是多少?

\begin{center}
\begin{tikzpicture}[>=latex, scale=.7]
\fill[pattern=north east lines, draw] (0,0) rectangle (4.5,4.5);
\draw [fill=white](1.2,1.2) rectangle (3.3,3.3);
\draw [<->](3.3,1.5)--node[fill=white, above]{2}(4.5,1.5);
\end{tikzpicture}
\end{center}

\item  老陈用13.10元钱,买笔记本和笔尖两种用品,笔记本
与笔尖的单价分别是1.50元与0.80元,你能知道老陈买了几
本笔记本?几个笔尖?
\item  一批任务,原计划每天要完成120件,实际上每天比原
计划多完成40件,结果提前6天完成任务.问:原计划几天
完成?这批任务一共多少件?
\item  加工一批零件,第一天完成$\frac{1}{2}$,第二天完成剩下的$\frac{1}{3}$,
这时只剩下18个零件没有加工.试问:第一、二两天各加工
了多少个零件?
\item  抽水机若干台,在规定时间内可以完成任务,如果减
少3台机器,就要延长6小时,如果增加2台机器,就能提
前2小时完成任务,问:原有几台抽水机?原计划几小时完
成所给任务?
\item  甲从$A$到$B$用8小时,乙从$B$到$A$用12小时,且甲比乙
每小时多走10里,如果甲、乙二人同时由$A, B$相向而行,经
过几小时,他们两人在距A多远的地方相遇?
\item  甲、乙、丙三人,甲每分钟走20丈,乙每分钟走22.5
丈,丙每分钟走25丈.如果甲、乙二人在东村,丙在西村,
他们三人同时相向而行.丙遇到乙以后三分钟才遇到甲.求
东、西两村之间的距离是多少丈?
\item  甲、乙二人同绕400米环形道赛跑,如果他们同时由
同一起点出发背向而跑,25秒相遇,如果他们同时由同一起
点同向而跑,2分零5秒甲能追上乙.试求甲、乙二人每秒
钟各跑几米?
\item  某市举行环城自行车赛.35分钟以后最快的运动员遇
到最慢的.已知环城一周为6公里,最慢的运动员的速度是
最快的速度的$\frac{5}{7}$.试求这两名运动员的速度各是多少?
\item  两个伐木队在一月份共伐木900立方米.二月份比一月
份第一队多伐15\%,第二队多伐12\%,因而二月份共伐木
1020立方米.试求二月份两队各伐木多少?
\item $A, B$两块合金,$A$重12公斤,含纯银70\%; $B$含纯银
56\%,现用两块合金熔化后得到含纯银60\%的新合金,求$B$
合金的重量.
\item 水池的进水管被堵,流量减少40,试间:注满一池
水所需要的时间增加了百分之几?
\item 由金、银、铜制成三种合金$A, B,C$;已知金、银、铜
的重量比在三种合金中分别是:在$A$中$1:3:2$,在$B$中
$2:1:1$;在$C$中$1:2:5$.而且知道三种合金中总共含金
5.5克,含银8克,含铜9.5克.试求:三种合金$A, B,C$各
自的重量.
\item 某车间每天可以生产甲种零件300个或乙种零件500个
或丙种零件600个.这三种零件各一个可以配成一套.现在
要在63天的生产中,便所生产的三种零件全部配套.试问:
这个车间应该对这三种零件的生产各用几天才成?
\item 小光用0.93元买了单价各为0.10元、0.12元、0.15元的
铅笔一共8支,如果把0.10元的铅笔数与0.12元的对调一
下,就可以节约0. 02元钱.问:小光原先买三种铅笔各多少
支?

\item (中国古代数学问题)
\begin{enumerate}
    \item 好马日行240里,劣马日行150里.今劣马先行12日,
    好马几日可追上劣马?
    \item 以绳测井.若将绳三折测之,绳多4尺,若将绳四
    折测之,绳多1尺.求绳长,井深各几何?
    \item 今卖牛二、羊五,可买猪十三,尚余钱一千;若卖
      牛三、猪三,恰可买羊九;若卖羊六,猪八,只能买牛五,
      还欠钱六百.试问:牛、羊、猪的单价各是多少?
\end{enumerate}


  \item 由海里提出40升海水,经化验知道其中含清水36.6
升;在这40升海水中又增添了若干清水后,再从中提出40
升,经化验后知道,这时含清水38升.试求:增添了多少清
  水?

\end{enumerate}

