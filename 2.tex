\section*{一次方程式}

一、这一章主要是应用数系运算通性及等式性
质,解一次方程和方程组,并且能解决相应的应用问
题.

\vskip 2ex 

    二、含有未知数的等式,叫做方程.能使方程式
两边相等的未知数的值,就是方程的解.

\vskip 2ex 

    三、含有一个未知数的方程,叫做一元方程,如
果只含有一个未知数,分母不含未知数,且最高次项
的指数是1的方程,叫做一元一次方程.解一元一次
方程的原理和方法是:由数的运算通性和等式性质,
归纳出去分母、去括号、移项变号、合并同类项、除
以未知数的系数等具体规则,应用这些具体规则,就
可以把方程的解求出来.

\vskip 2ex 

    四、含有两个未知数的一次方程,叫做二元一次
方程.二元一次方程的解是一个数值组,记作$(x, y)$;
任一个二元一次方程的解都有无限多个.

\vskip 2ex 

    五、两个以上的方程联合在一起,组成一个方程
组.因而,由几个含有两个未知数的一次方程所组成
的方程组二网做三元一次方程组.能够同时满足方程
组中所有各个方程的未知数所取的数值组,叫做这个
方程组的解.

\vskip 2ex 

    六、解多元一次方程组的关键是消元.具体方法
有:加、减消元法和代入消元法.

\vskip 2ex 

    这里应特别指出:加减法消元是较普遍且重要的
方法,其要点就是,应用等式的性质,将两个方程中
相同的某一个未知数的系数变形成为绝对值相等.然
后把方程的两边分别相加或相减,就可以消去这个未
知数.

    因此,可以说:解多元一次方程的过程,就是逐
步消元求解的过程.即
\begin{center}
\begin{tikzpicture}[>=latex, scale=1.3]
\node  (A) at (0,0) {\Large 多元} ;
\node  (B) at (2,0) {\Large ……} ;
\node  (C) at (4,0) {\Large 三元} ;
\node  (D) at (6,0) {\Large 二元} ;
\node  (E) at (8,0) {\Large 一元} ;

\draw[->] (A)--node[above]{消元}(B);
\draw[->] (B)--node[above]{消元}(C);
\draw[->] (C)--node[above]{消元}(D);
\draw[->] (D)--node[above]{消元}(E);

\end{tikzpicture}
\end{center}

\vskip 2ex 

七、解应用问题就是运用数学工具解决实际问
题,这就要:
\begin{enumerate}
    \item 审题:弄清题意,分析问题中涉及到的量与量
    之间的关系.
    \item 引入未知数,并用未知数表示出有关的量.
    \item 正确列出方程(或方程组).一般来说,引入
    的未知数个数与所列方程的个数是相等的.
    \item 准确地求出来知数的值,即所列方程(方程
    组)的解.
    \item 检验:所列方程(方程组)的解是否符合题
    意.将不合理的值舍去,从而写出正确答案.
\end{enumerate}

        其中,分析量之间的关系是列方程的关键;列方
    程是解决问题的基础;解方程又是解决问题的主要手
    段.这都是解应用问题时,必须注意的几个重要环节.