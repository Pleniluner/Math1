\section*{多项式的四则运算}

这一章的主要内容是多项式的有关概念及其四则运算.

一、由己知数与未知数符号的方幂相乘而得到的
式子叫单项式;若干个单项式的代数和,叫做多项式.多项式又叫做整式.

多项式的次数,就是指多项式中,次数最高的某一单项式的次数;而单项式的次数,就是所含各未知数的指数和.

任一非零常数,叫做零次多项式.

数零,叫做零多项式,它的次数不定.

\vskip 2ex 

二、一元$n$次多项式$f(x)$的标准形式是:
\[f (x) =a_n x^n +a_{n-1}x^{n-1}+\cdots+a_1x+a_0,\qquad (a_0\ne 0) \]
当$x=b$时,$f(x)$的值记作:
\[f (b) =a_n b^n +a_{n-1}b^{n-1}+\cdots +a_1b+a_0\]
\vskip 2ex 

三、两个多项式$f(x)$与$g(x)$,当$x$取任意值时,它们的值总相等,那么这两个多项式称为恒等,记作:
\[f (x) \equiv g (x) \]
如果$f(x)\equiv g(x)$,那么,这两个多项式相应的
同类项的系数都相等.这是待定系数法的依据.
\vskip 2ex 

四、能够使多项式$f(x)=0$的$x$的值,叫做多项式
$f(x)$的根.要求多项式$f(x)$的根,只要解方程$f(x)=0$就可以得到.
\vskip 2ex 

五、多项式的加、减,乘法运算的结果仍是多项
式(封闭的).而且具有数系运算的通性.
\vskip 2ex 

如果用$f,g$表示两个多项式(一元或多元),它
们分别为$m$次和$n$次($m>n$).那么,
\begin{itemize}
    \item $f+g$是一个$m$次多项式,$f-g$也是$m$次多项
    式;而$f\cdot g$是一个$(m+n)$次多项式.
\end{itemize}
\vskip 2ex 

六、常用乘法公式是运算的工具,必须掌握:
\[\begin{split}
    (a-b)(a+b)&= a^2-b^2\\
    (a\pm b)^2&=a^2\pm 2ab+b^2\\
    (a\pm b)^3&=a^3\pm 3a^2b+3ab^2\pm b^3\\
    (a\pm b)(a^2\mp ab+b^2)&=a^3\pm b^3
\end{split}\]

这些公式中的$a,b$,可以是数、或单项式、或多项式.因此,在应用中要具体分析,灵活掌握.
\vskip 2ex 

七、带余除法,是一元多项式的特有运算.

\begin{itemize}
    \item $f(x)$除以$g(x)$,就是要求出两个多项式$Q(x)$
    与$R(x)$,使它们满足关系式:
    \[f (x) =Q (x) \cdot g (x) +R (x)\]
    其中,$R(x)$的次数要低于$g(x)$的次数.
    \item 除法的原理是:逐步寻求单项式,进行降次
    工作.具体方法有:类似于整数除法的长除法、分离系数法、待定系数法以及除式为一次式的综合除法.
    \item 两个一元多项式相除,当除式乘以一个非零常数$k$时,所得商式就是原商式的$\frac{1}{k}$,所得余式不变,这是由于:
\[\begin{split}
    f(x)&=Q(x)\cdot g(x)+R(x)\\
    &=\frac{1}{k}Q'(x)\cdot [k\cdot g(x)]+R'(x)
\end{split}\]    

\end{itemize}