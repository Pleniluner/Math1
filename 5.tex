\section*{因式分解与余式定理}

这一章是第四章多项式理论的继续和深入,主要内容有:因式分解、余式定理及其推论和应用、辗转相除法及其应用.

一、在指定范围内,把一个多项式写成几个次数
较低的不可约多项式之积的变形,就是多项式的因式分解.

多项式因式分解的常见方法有:
\begin{enumerate}
    \item 提取公因式法;
    \item 分组分解法;
    \item 乘法公式分解法;
    \item 配方法、视察法分解二次三项式;
    \item 待定系数法分解二元二次多项式.
\end{enumerate} 

二、余式定理是:多项式$f(x)$除以$(x-a)$所得
的余式是$f(a)$,即
\[f (x) =q (x) \cdot (x-a) +f (a) \]
由此,可以得到以下推论:
\begin{enumerate}
\item 如果$f(a)=0$(或说$a$为$f(x)$的根),那么,$f(x)$
可以被$(x-a)$整除.反过来也正确.
\item 如果$f(a)=0,f(b)=0$(或说$a,b$为$f(x)$的两个不同的根),那么,$f(x)$必可以被$(x-a)(x-b)$整除,也就是$f(x)$必含有因式$(x-a)(x-b)$.反过来说,也是正确的.
\item 一元$n$次多项式$f(x)$,至多只能有$n$个不同的根.
\item  如果:已知$f(a_1),f(a_2),f(a_3),\ldots,f(a_{n+1})$
共$n+1$个值,那么,就可以确定一个$n$次多项式.
\end{enumerate}

三、综合运用余式定理及其推论、综合除法及待
定系数法,可以进行因式分解、求整系数多项式的有理根.

在这里,可以进一步发现,解方程与因式分解是互通的.因为:
如果$f(x)$可以分解为$(mx+n)\cdot (px+q)\cdots (rx+s)$,那么,方程$f(x)=0$就一定有有理根$-\frac{n}{m},-\frac{q}{p},\ldots,-\frac{s}{r}$.

反过来,如果方程$f(x)=0$有有理根$-\frac{b}{a},-\frac{d}{c},\ldots,-\frac{f}{e}$.那么,多项式$f(x)$就一定含有因式:$(ax+b)(cx+d)\cdots(ex+f)$.

四、辗转相除法

利用辗转相除法,可以求出几个多项式的最高公因式和最低公倍式.

两个多项式$f(x)$与$g(x)$的最高公因式记为:$(f(x),g(x))$;最低公倍式记为:$[f(x), g(x)]$.由于它们都不计非零常数因子,因而有以下关系式:
\[ kf (x) g (x) = (f(x), g(x))\cdot [f(x),g(x)]\]
\[[f(x),g(x)]=\frac{f(x)\cdot g(x)}{(f(x), g(x))}\qquad \text{(非零常数$k$不计)}\]
