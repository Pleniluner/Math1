\section*{代数运算的初步应用}}

本章的主要内容是两种常见数列的求和及待定系数法与它的应用.

一、等差数列

\begin{enumerate}
    \item 按顺序排好的一列数中,如果从第二个数
起,每一个数与它前一个数的差都相等,那么,这一列数叫做等差数列.

设等差数列的首项为$a_1$, 公差为$d$, 项数为$n$, 末项为$a_n$及前$n$项和为$S_n$, 则有以下关系式:
\[\begin{split}
    a_n&=a_1+(n-1)d\\
    S_n&=\frac{n}{2}[2a_1+(n-1)d]=\frac{n}{2}(a_1+a_n)
\end{split}\]

如果已知$a_1,a_n,n,d,S_n$中的任意三个,就可以利用这两个公式,求出另两个.

\item 在两个已知数$a,b$之间,插入$n$个数构成等
差数列的问题,实际上就是已知首项$a$, 末项$b$及项数$n+2$, 要求出公差,进而可以求出插入的各项,还可以求出所有项的和.
\end{enumerate}

\vskip 2ex 
二、等比数列

按顺序排好的一列数中,如果从第二个数起,每一个数与它前一个数的比都相等,那么,这一列数叫
做等比数列.

设等比数列的首项为$a_1$, 公比为$q$, 项数为$n$, 末项为$a_n$, 前$n$项的和为$S_n$, 则有以下关系式:
\[\begin{split}
    a_n&=a_1q^{n-1}\\
    S_n&=\frac{a_1(1-q^n)}{1-q}
\end{split}\]

公比$q$的取值,决定了等比数列各项的大小变化趋向:如果首项$a_1>0$(或$<0$),那么,
\begin{itemize}
    \item 当$q>1$时,等比数列逐项增大(或减小);
    \item 当$0<q<1$时,等比数列逐项减小(或增大);
    \item 当$q<0$时,等比数列各项将正、负相间,逐项在正、负值之间摆动.
\end{itemize}

\vskip 2ex 
三、待定系数法是一个重要的数学方法,其根据就是多项式恒等的性质.其方法的要点就是:引进未定系数,列出恒等式并进而得出含有未定系数的方程组,求出未定系数.

待定系数法应用广泛,具体作法中又有一定的技巧,除可以求商式、余式、分解因式、寻求方程的根与系数间的关系外,还应从以下应用中进一步去掌握:
\begin{itemize}
    \item 求多项式与解方程;
    \item    用一个较低次的多项式的各次幂,表示另一个多项式;
    \item     将分式化成部分分式;
    \item 求形如$a\pm 2\sqrt{b}$的数的算术平方根;
    \item 求数列$1^b,\; 2^b,\; 3^b, \ldots,n^b,\ldots$ 前$n$项和($b$是大于1的一个自然数).
\end{itemize}