\section*{一元二次方程}
一、学习一元二次方程所必要的预备知识和概
念:
\begin{enumerate}
    \item 完全平方公式:$(x\pm y)^2=x^2\pm 2xy+y^2$或
称为二数和(或差)的平方公式.当$x,y$取正数时,
这两个公式都可以用“画正方形”的方法加以验证.
\item 平方根的概念、性质及运算.

如果$x^2=a,\; (a\ge 0)$,那么$x$就是$a$的平方根,当
$a>0$时,它有两个平方根,记作$x=\pm\sqrt{a}$.当$a=0$
时,它有一个平方根,记作
$x=\sqrt{0}=0$.当$a<0$
时,它的平方根无意义.

正数的正平方根,叫做它的算术平方根,记作:
$x=\sqrt{a}\; (a>0)$.

零的算术根,仍然是0.

对于算术平方根
$x=\sqrt{a}\; (a\ge 0)$,有以下基本性
质:
\begin{enumerate}
    \item $(\sqrt{a})^2=a\qquad (a\ge 0)$
    \item $\sqrt{x^2}=|x|=\begin{cases}
        x& x\ge 0\\
        -x& x<0
    \end{cases}$
\end{enumerate}


算术平方根的运算法则:
\begin{enumerate}
    \item $\sqrt{a}\cdot \sqrt{b}=\sqrt{ab}\quad (a\ge 0,\; b\ge 0)$
    
    运用它,可以进行“因数移到根号里”和“因数
移到根号外”的变形.

\item $\frac{\sqrt{a}}{\sqrt{b}}=\sqrt{\frac{a}{b}}\quad (a\ge 0,\; b> 0)$

运用分数的基本性质及算术平方根的基本性质,
可以进行“有理化分母”的变形.

\item 把算术平方根化简(使被开方数的每个因数的
指数小于2;分母有理化)以后,相同被开方数的算
术根可以运用数系运算通性进行加、减运算.
\end{enumerate}

\item 求一个数的平方根,可以查平方根表,也可以
直接开平方,进行计算.
\item 实数.

无限不循环小数,称为无理数.例如:$\sqrt{2},\; -\sqrt{3},\; \pi,\; e,\; 0.01001000\cdots$,等都是无理数.

无理数与有理数,统称为实数.

任一个非零实数$a$, 都有一个相反数$-a$, 且满足
$a+(-a)=0$; 都有一个倒数$\frac{1}{a}$,
且满足$a\cdot \frac{1}{a}=1$.

实数的运算,同样具有运算通性.
\end{enumerate}

\vskip 2ex 
二、一元二次方程的标准式是
$$ax^2+bx+c=0\qquad (a\ne 0)$$
它的解法有:

\begin{enumerate}
    \item 配方法:关键是先化为二次项系数为1的形
式,然后将方程“两边加上一次项系数一半的平方
数”,使方程一边成为完全平方形式.
\[\begin{split}
    x^2+\frac{b}{a}x+\frac{c}{a}&=0\\
    x^2+\frac{b}{a}x+\left(\frac{b}{2a}\right)^2&=\left(\frac{b}{2a}\right)^2-\frac{c}{a}\\
    \left(x+\frac{b}{2a}\right)^2&=\frac{b^2-4ac}{4a^2}
\end{split}\]
\begin{equation}
    x=\frac{-b\pm\sqrt{b^2-4ac}}{2a}
\end{equation}

\item 公式法——把方程化为标准式,找出各项系数
$a,b,c$,代入(3.4)就可以求出根.
\item 换元法——如果能把方程整理成如下形式
\[a(x+m)^2+b(x+m)+c=0\qquad (a\ne 0)\]
那么,可以设$(x+m)=y$, 原方程化为
\[ay^2+by+c=0\qquad  (a\ne 0)\]
用求根公式先求出$y$的值,然后再代入原设:$(x+m)
=y$, 进而求出原方程的根.
\item 对于$b=0$, 或$c=0$, 或$b=c=0$时的特殊一元
二次方程,除以上一般解法外,还可以直接运用数系
运算通性(特别是分配律)、平方根的意义等方法,
求出方程的根.
\end{enumerate}

\vskip 2ex 
三、利用一元二次方程,还可以解一些特殊的高
次方程,其主要方程是换元法——设辅助未知数.通
过换元,把高次转化为低次方程,从而可以由已知解
法的低次方程,逐步达到求出未知的高次方程的根.
本章我们所遇到过的有以下几种形式:
\begin{enumerate}
    \item $ ax^4+bx^2+c=0\qquad  (a\ne 0)$
    
    可设$x^2=y$, 原方程化为$ay^2+by+c=0$.
    \item $a(mx^2+nx)^2+b(mx^2+nx)+c=0\qquad  (a\ne 0)$
    
    可设$mx^2+nx=y$, 原方程化为$ay^2+by+c=0$.
    \item $a(mx^2+nx+p)(mx^2+nx+q)+b=0$
    
    可设$mx^2+nx=y$, 原方程化为$a(y+p)(y+q)+b=0$.
    \item $ax^3+bx^2+cx=0\qquad (a\ne 0)$
    
    用分配律:$(ax^2+bx+c)x=0$,把原方程可以
化为两个低次方程求解,即:
$ax^2+bx+c=0$ 或 $x=0$
\end{enumerate}

\vskip 2ex 
四、一元二次方程根的判别式:
\[\Delta =b^2-4ac\]
这是判别任一个一元二次方程的根是否存在以及存在
什么样的实数根的准绳.即
\begin{enumerate}
\item 当$\Delta>0$时,$ax^2+bx+c=0$有两不等实根.
\item 当$\Delta=0$时,$ax^2+bx+c=0$有两相等实根(重
根).
\item 当$\Delta<0$时,$ax^2+bx+c=0$无实数根.
\end{enumerate}

五、用一元二次方程解应用问题,其方法与主要
步骤与第二章一次方程解应用题是一样的.其要点仍
是:弄清题意,分析量与量之间的关系,引入未知
数,列出方程式,这是解决问题的基础;进而解方
程,得到实数解(或无解).这是解决问题的关键;
最后,还应检验所得解是否符合题意.舍去不合理
的,留下符合题意的,写出答案.


