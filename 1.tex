\section*{有理数系}

一、复习和总结了小学算术中所学的数(自然
数,零及正分数)和四则运算,并进一步明确了自然
数的意义,系统说明了运算中的普遍性质(通性)这
就是:加法交换、结合律;乘法交换、结合律;分配
律及0、1的运算特性,还引进了乘方运算和指数运
算律:

乘方运算:相同因数的连乘积.
\[a^n=\underbrace{a\cdot a\cdot a\cdots a}_{\text{$n$个$a$}},\qquad a^1=a \]

指数运算律及零指数的意义:
\[a^m\cdot a^n=a^{m+n},\qquad (a\cdot b)^n=a^n\cdot b^n,\qquad (a^m)^n=a^{m\cdot n} \]
\[\left(\frac{a}{b}\right)^n=\frac{a^n}{b^n}\; (b\ne 0),\qquad a^m\div a^n=a^{m-n}\; (a\ne 0,\; m\ge n),\qquad a^0=1\;(a\ne 0) \]

\vskip 2ex 

二、由相反意义的量,引进了意义相反的正数与
负数,其特征就是“合并时,可以相消或部分相消”.
特别地,$(-a) +(+a) = 0$时,$-a$与$a$叫做互为相反
的数.$+1$与$-1$是相反意义的单位.因而将数的范围
就扩大到有理数.

有理数包括正、负整数,0及正、负分数.

一切有理数,组成有理数集合,它包含了整数集
合,而整数集合又包含了自然数集合.

任一个有理数,都可以用数轴上的一个点表示出
来.

\vskip 2ex 

三、有理数的运算法则

对有理数的运算法则,我们都是在承认运算通性
(包括运算律、指数运算律,0与1的特性)仍然有
效的前提下,合理地作了规定的.
\begin{enumerate}
	\item 加法法则:设$a,  b$是正有理数.
	\[\begin{split}
	(+a)+(+b)&=+(a+b)\\
	(-a)+(-b)&=-(a+b)\\
	(+a)+(-b)&=\begin{cases}
	+(a-b) & (a>b)\\  
	-(b-a) & (a<b)\\  
	\end{cases}    \\
	(+a)+0&=0+(+a)=+a\\
	(-a)+0&=0+(-a)=-a\\
	0+0&=0
	\end{split}\]
	\item 减法法则:设$x,  y$是有理数,
	$x-y=x+(-y)$
	\item 乘法法则:设$a,  b$是正有理数,
	\[\begin{split}
	(+a)\cdot (+b)&=+ab\\
	(-a)\cdot (+b)&=(+a)\cdot (-b)=-ab\\
	(-a)\cdot (-b)&=+ab \\
	(+a)\cdot 0&=(-a)\cdot 0=0\\
	0\x 0&=0
	\end{split}\]
	
	\item 除法法则:设$x, y$是有理数,$y\ne 0$.
	\[x\div y=x\x\frac{1}{y} \]
	
	\item 乘方法则:除了有效地应用指数运算律之外,
	有理数乘方还有以下规则:
	\begin{itemize}
		\item 正数的任何次方,仍是正数.
		\item 零的非零次方,仍是零.
		\item 负数的偶次方是正数;而负数的奇次方是负
		数.
	\end{itemize}
	
\end{enumerate}




\vskip 2ex 



四、有理数系是一个运算简便易行,通行无阻的
数集.是今后讨论数量的实际问题的有力工具.有理
数的性质可以概括为:
\begin{enumerate}
	\item 运算性质(通性).
	
	四则运算封闭;加法、乘法的交换律、结合律和
	分配律成立;零、1在运算中的特性;五个指数运算
	律成立;$a^0=1,\; (a\ne 0)$.
	
	\item 有理数可以比大小.
	\begin{itemize}
		\item 当$a-b>0$时,$a>b$;
		\item  当$a-b=0$时,$a=b$;
		\item 当$a-b<0$时,$a<b$.
	\end{itemize}
	而且,有理数的大小,正好相当于在数轴上表示这些
	有理数的点的右、左位置.我们称为“有理数的有序
	性”.
	\item 有理数的稠密性:任意两个有理数之间,存在
	着很多有理数.
\end{enumerate}

\vskip 2ex 

五、等式与不等式的基本性质.


