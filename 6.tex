\section*{分式与根式}

本章是在学习多项式(整式)的基础上,进一步学习了分式及其运算、根式及其运算和分式方程、根式方程.

一、多项式,分式,根式等,都是含有数字和字
母并涉及加、减、乘、除、乘方、开平方六种代数运算的式子,这些式子统称为代数式.其中,凡只涉及字母的加、减、乘、乘方运算的式子,叫做多项式
(整式);凡涉及字母的除法运算且字母含在除式中的式子,叫做分式;
分式与整式,又统称为有理式.

凡涉及字母或数字的开方运算的式子叫做根式,根号内含有字母的根式,又称为关于这个字母的无理
式.如$\sqrt{2}$是根式,但不是无理式,$\sqrt{x}$,$\sqrt{x-1}$,$\frac{x}{\sqrt{x-2}}$
等都是无理式.

关于代数式的概念,可以列表如下:
\[
\text{代数式}\begin{cases}
    \text{有理式} & \begin{cases}
        \text{多项式(即整式)}\\\text{分式}
    \end{cases}\\
    \text{无理式}
\end{cases}    
\]

二、如果有多项式$f(x),g(x)$且$g(x)$的次数大于
零次,那么分式$\frac{f (x)}{g (x)}$有以下基本性质:
\[\frac{f (x) \cdot h (x)}{g(x)\cdot h(x)}=\frac{f(x)}{g (x)}=\frac{f (x) \div h (x)}{g(x)\div h(x)}\]
其中,$h(x)$是非零多项式.

利用基本性质,可以进行分式的通分和约分.分式的四则运算和分数的四则运算是一样的.

三、表示平方根的式子,叫做二次根式.

二次根式有以下基本性质:
\begin{itemize}
    \item $\left(\sqrt{a}\right)^2=1\quad (a\ge 0)$
    \item $\sqrt{a^2}=|a|$
    \item $\sqrt{ab}=\sqrt{a}\cdot \sqrt{b}\quad (a\ge 0,\; b\ge 0)$
    \item $\sqrt{a\div b}=\sqrt{a}\div \sqrt{b} \quad (a\ge 0,\; b>0)$
\end{itemize}

利用基本性质,二次根式可以进行以下变形:
\begin{enumerate}
    \item 因式的内移与外移,即
    \[\begin{split}
        m\sqrt{a}&=\sqrt{am^2}\quad (m>0)\\
        \sqrt{a^2m}&=a\sqrt{m}\quad (a>0)
    \end{split}\]
    \item 化去根号内的分母或化去分母中的根号——都是有理化分母的内容,即
\[\sqrt{\frac{a}{b}}=\sqrt{\frac{a\x b}{b\x b}}=\frac{\sqrt{ab}}{b}\qquad (b>0)\]
或
\[\sqrt{\frac{a}{b}}=\frac{\sqrt{a}}{\sqrt{b}}=\frac{\sqrt{a}\cdot \sqrt{b}}{\sqrt{b}\cdot \sqrt{b}}=\frac{\sqrt{ab}}{b}\qquad (b>0)\]
\end{enumerate}

如果一个二次根式符合条件:
\begin{enumerate}
    \item 被开方各因数的指数小于2;
    \item 根号内不含分母(即分母已经有理化).
\end{enumerate}
那么这个二次根式就叫做最简二次根式.

如果几个二次根式化为最简根式以后,根号内的式子相同,那么,这几个二次根式就叫做同类根式.同类根式和同类项一样可以合并.

二次根式的四则运算和多项式的运算很类似.只
要注意化为最简根式和合并同类根式就行了.

四、分式方程与根式方程的解法要点是:设法转
化为整式方程求解.由于它们的特点不同,转化方法也就不同.

分式方程的特点是:分母中含有未知数.因而,要利用分式的基本性质或等式的基本性质,两边乘以同一个整式(一般是取各分母的最低公倍式),约简后
转化为一个整式方程.

根式方程的特点是:根号内含有未知数,因而,就要方程两边同次乘方(如:同平方)后,利用根式的基本性质转化为有理方程,并进而转化为整式方程.

但是,一定要注意,在解分式方程与根式方程的过程中,由于各自的转化方式都能引起未知数允许取值范围的扩大,所以都可能产生增根.因此,无论是解分式方程,还是解根式方程,最后的验根都是不可缺少的,验根,将起到“识别真假”“去伪存真”的作用.

五、已学过的方程有:整式方程、分式方程、根
式方程,统称为代数方程.其系统可列表如下:
\begin{center}
\begin{tikzpicture}[yscale=.6]
    \node at (6,3)[right]{一次方程};
    \node at (6,2)[right]{二次方程};
    \node at (6,1)[right]{高次方程};
    \node at (4,2)[right]{整式方程};
    \node at (4,0)[right]{分式方程};
    \node at (2,1)[right]{有理方程};
    \node at (2,-1)[right]{根式方程};
    \node at (0,0)[right]{代数方程};
    \draw[decorate,decoration=brace, thick](6,1)--(6,3);
    \draw[decorate,decoration=brace, thick](4,0)--(4,2);
    \draw[decorate,decoration=brace, thick](2,-1)--(2,1);
\end{tikzpicture}
\end{center}
